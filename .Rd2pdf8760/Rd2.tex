\documentclass[a4paper]{book}
\usepackage[times,inconsolata,hyper]{Rd}
\usepackage{makeidx}
\usepackage[latin1]{inputenc} % @SET ENCODING@
% \usepackage{graphicx} % @USE GRAPHICX@
\makeindex{}
\begin{document}
\chapter*{}
\begin{center}
{\textbf{\huge \R{} documentation}} \par\bigskip{{\Large of \file{man/Actif.Rd} etc.}}
\par\bigskip{\large \today}
\end{center}
\inputencoding{utf8}
\HeaderA{Actif}{Classe \code{Actif}}{Actif}
\keyword{classes}{Actif}
%
\begin{Description}\relax
Cette classe aggrege l'ensemble des donnees relatives a l'actif de la compagnie d'assurance : portefeuille, hypotheses
\end{Description}
%
\begin{Section}{Slots}

\begin{description}

\item[\code{ptf\_actif}] est un objet de la classe \code{\LinkA{PTFActif}{PTFActif}} representant le portfeuille financier.

\item[\code{hyp\_actif}] est un objet de la classe \code{\LinkA{HypActif}{HypActif}} representant les hypotheses du portfeuille financier.

\end{description}
\end{Section}
%
\begin{Author}\relax
Damien Tichit pour Sia Partners
\end{Author}
\inputencoding{utf8}
\HeaderA{Action}{Classe \code{Action}}{Action}
\keyword{classes}{Action}
%
\begin{Description}\relax
Cette classe represente le portefeuille des actions de la compagnie d'assurance.
\end{Description}
%
\begin{Section}{Slots}

\begin{description}

\item[\code{ptf}] est un objet de type \code{\LinkA{data.frame}{data.frame}} contenant les donnees relatives au portefeuille.

\end{description}
\end{Section}
%
\begin{Author}\relax
Damien Tichit pour Sia Partners
\end{Author}
\inputencoding{utf8}
\HeaderA{aggregation\_actif}{Fonction \code{aggregation\_actif}.}{aggregation.Rul.actif}
%
\begin{Description}\relax
Cette fonction permet de faire appel aux diffC)rentes fonctions permettant d'aggreger les portfeuilles financiers
d'une compagnie d'assurance.
\end{Description}
%
\begin{Usage}
\begin{verbatim}
aggregation_actif(actif)
\end{verbatim}
\end{Usage}
%
\begin{Arguments}
\begin{ldescription}
\item[\code{actif}] est un objet de type \code{\LinkA{Actif}{Actif}}.
\end{ldescription}
\end{Arguments}
%
\begin{Author}\relax
Damien Tichit pour Sia Partners
\end{Author}
\inputencoding{utf8}
\HeaderA{aggregation\_action}{Fonction \code{aggregation\_action}.}{aggregation.Rul.action}
%
\begin{Description}\relax
Cette fonction permet d'aggreger les model-point pour un portfeuille d'actions.
\end{Description}
%
\begin{Usage}
\begin{verbatim}
aggregation_action(action)
\end{verbatim}
\end{Usage}
%
\begin{Arguments}
\begin{ldescription}
\item[\code{action}] est un objet de type \code{\LinkA{Action}{Action}}.
\end{ldescription}
\end{Arguments}
%
\begin{Author}\relax
Damien Tichit pour Sia Partners
\end{Author}
\inputencoding{utf8}
\HeaderA{aggregation\_alm}{Fonction \code{aggregation\_alm}.}{aggregation.Rul.alm}
%
\begin{Description}\relax
Cette fonction permet de faire appel aux diffC)rentes fonctions permettant d'aggreger les portefeuilles.
\end{Description}
%
\begin{Usage}
\begin{verbatim}
aggregation_alm(alm)
\end{verbatim}
\end{Usage}
%
\begin{Arguments}
\begin{ldescription}
\item[\code{alm}] est un objet de type \code{\LinkA{ALM}{ALM}}.
\end{ldescription}
\end{Arguments}
%
\begin{Author}\relax
Damien Tichit pour Sia Partners
\end{Author}
\inputencoding{utf8}
\HeaderA{aggregation\_epargne\_1}{Fonction \code{aggregation\_epargne\_1}.}{aggregation.Rul.epargne.Rul.1}
%
\begin{Description}\relax
Cette fonction permet d'effectuer une premiere aggregation des model-point pour un portefeuille de contrats d'epargne.
Cette aggregation se fait par tmg, sexe, age et anciennete. Elle s'effectue avant le calcul des probas.
\end{Description}
%
\begin{Usage}
\begin{verbatim}
aggregation_epargne_1(epargne)
\end{verbatim}
\end{Usage}
%
\begin{Arguments}
\begin{ldescription}
\item[\code{epargne}] est un objet de type \code{\LinkA{Epargne}{Epargne}}.
\end{ldescription}
\end{Arguments}
%
\begin{Author}\relax
Damien Tichit pour Sia Partners
\end{Author}
\inputencoding{utf8}
\HeaderA{aggregation\_epargne\_2}{Fonction \code{aggregation\_epargne\_2}.}{aggregation.Rul.epargne.Rul.2}
%
\begin{Description}\relax
Cette fonction permet d'effectuer la derniere aggregation des model-point pour un portefeuille de contrats d'epargne.
Cette aggregation se fait uniquement par tmg. Elle s'effectue apres le calcul des probas.
\end{Description}
%
\begin{Usage}
\begin{verbatim}
aggregation_epargne_2(epargne)
\end{verbatim}
\end{Usage}
%
\begin{Arguments}
\begin{ldescription}
\item[\code{epargne}] est un objet de type \code{\LinkA{Epargne}{Epargne}}.
\end{ldescription}
\end{Arguments}
%
\begin{Author}\relax
Damien Tichit pour Sia Partners
\end{Author}
\inputencoding{utf8}
\HeaderA{aggregation\_immobilier}{Fonction \code{aggregation\_immobilier}.}{aggregation.Rul.immobilier}
%
\begin{Description}\relax
Cette fonction permet d'aggreger les model-point pour un portfeuille d'actions.
\end{Description}
%
\begin{Usage}
\begin{verbatim}
aggregation_immobilier(immobilier)
\end{verbatim}
\end{Usage}
%
\begin{Arguments}
\begin{ldescription}
\item[\code{immobilier}] est un objet de type \code{\LinkA{Immobilier}{Immobilier}}.
\end{ldescription}
\end{Arguments}
%
\begin{Author}\relax
Damien Tichit pour Sia Partners
\end{Author}
\inputencoding{utf8}
\HeaderA{aggregation\_obligation}{Fonction \code{aggregation\_obligation}.}{aggregation.Rul.obligation}
%
\begin{Description}\relax
Cette fonction permet d'aggreger les model-point pour un portfeuille d'obligations.
\end{Description}
%
\begin{Usage}
\begin{verbatim}
aggregation_obligation(obligation)
\end{verbatim}
\end{Usage}
%
\begin{Arguments}
\begin{ldescription}
\item[\code{obligation}] est un objet de type \code{\LinkA{Obligation}{Obligation}}.
\end{ldescription}
\end{Arguments}
%
\begin{Author}\relax
Damien Tichit pour Sia Partners
\end{Author}
\inputencoding{utf8}
\HeaderA{aggregation\_passif\_1}{Fonction \code{aggregation\_passif\_1}.}{aggregation.Rul.passif.Rul.1}
%
\begin{Description}\relax
Cette fonction permet de faire appel aux diffC)rentes fonctions permettant d'aggreger les portfeuilles passifs d'une compagnie d'assurance.
\end{Description}
%
\begin{Usage}
\begin{verbatim}
aggregation_passif_1(passif)
\end{verbatim}
\end{Usage}
%
\begin{Arguments}
\begin{ldescription}
\item[\code{passif}] est un objet de type \code{\LinkA{Passif}{Passif}}.
\end{ldescription}
\end{Arguments}
%
\begin{Author}\relax
Damien Tichit pour Sia Partners
\end{Author}
\inputencoding{utf8}
\HeaderA{aggregation\_passif\_2}{Fonction \code{aggregation\_passif\_2}.}{aggregation.Rul.passif.Rul.2}
%
\begin{Description}\relax
Cette fonction permet d'effectuer une premiere aggregation des portfeuilles passifs d'une compagnie d'assurance.
\end{Description}
%
\begin{Usage}
\begin{verbatim}
aggregation_passif_2(passif)
\end{verbatim}
\end{Usage}
%
\begin{Arguments}
\begin{ldescription}
\item[\code{passif}] est un objet de type \code{\LinkA{Passif}{Passif}}.
\end{ldescription}
\end{Arguments}
%
\begin{Author}\relax
Damien Tichit pour Sia Partners
\end{Author}
\inputencoding{utf8}
\HeaderA{agregation\_proba}{Fonction \code{agregation\_proba}.}{agregation.Rul.proba}
%
\begin{Description}\relax
Cette fonction permet d'agreger les probas par PM ou par nombre de contrats.
\end{Description}
%
\begin{Usage}
\begin{verbatim}
agregation_proba(epargne, proba, variable)
\end{verbatim}
\end{Usage}
%
\begin{Arguments}
\begin{ldescription}
\item[\code{epargne}] est un objet de type \code{\LinkA{Epargne}{Epargne}}.

\item[\code{proba}] est un \code{data.frame} contenant les probabilites devant etres agregees.

\item[\code{variable}] est un \code{character}. Elle indique selon quelle variable l'agregation doit etre faite.
\end{ldescription}
\end{Arguments}
%
\begin{Author}\relax
Damien Tichit pour Sia Partners
\end{Author}
\inputencoding{utf8}
\HeaderA{allocation\_ptf\_actif}{Fonction \code{allocation\_ptf\_actif}}{allocation.Rul.ptf.Rul.actif}
%
\begin{Description}\relax
Cette fonction permet de determiner l'allocation des differents produits financiers.
\end{Description}
%
\begin{Usage}
\begin{verbatim}
allocation_ptf_actif(ptf_actif)
\end{verbatim}
\end{Usage}
%
\begin{Arguments}
\begin{ldescription}
\item[\code{ptf\_actif}] est un objet de type \code{\LinkA{PTFActif}{PTFActif}}.
\end{ldescription}
\end{Arguments}
%
\begin{Author}\relax
Damien Tichit pour Sia Partners
\end{Author}
\inputencoding{utf8}
\HeaderA{allocation\_vnc\_ptf\_actif}{Fonction \code{allocation\_vnc\_ptf\_actif}}{allocation.Rul.vnc.Rul.ptf.Rul.actif}
%
\begin{Description}\relax
Cette fonction permet de determiner la VNC des differents produits financiers.
\end{Description}
%
\begin{Usage}
\begin{verbatim}
allocation_vnc_ptf_actif(ptf_actif)
\end{verbatim}
\end{Usage}
%
\begin{Arguments}
\begin{ldescription}
\item[\code{ptf\_actif}] est un objet de type \code{PTFActif}.
\end{ldescription}
\end{Arguments}
%
\begin{Author}\relax
Damien Tichit pour Sia Partners
\end{Author}
\inputencoding{utf8}
\HeaderA{ALM}{Classe \code{ALM}}{ALM}
\keyword{classes}{ALM}
%
\begin{Description}\relax
Cette classe aggrege un objet \code{\LinkA{System}{System}} ainsi que l'ensemble des hypotheses du modele ALM.
\end{Description}
%
\begin{Section}{Slots}

\begin{description}

\item[\code{system}] est un objet de la classe \code{\LinkA{System}{System}} qui aggregent les actifs et les passifs ainsi que leurs hypothC(ses.

\item[\code{hyp\_alm}] est un objet de la classe \code{\LinkA{HypALM}{HypALM}} qui contient les differentes hypotheses du modele ALM.

\end{description}
\end{Section}
%
\begin{Author}\relax
Damien Tichit pour Sia Partners
\end{Author}
\inputencoding{utf8}
\HeaderA{besoin\_revalo\_epargne}{Fonction \code{besoin\_revalo\_epargne}}{besoin.Rul.revalo.Rul.epargne}
%
\begin{Description}\relax
Cette fonction permet de calculer les besoins en revalorisation :
\begin{description}

\item[besoins contractuels]  : besoins pour assouvir les engagements contractuels, a savoir le tmg.
\item[besoins cibles]  : besoins pour atteindre une revalorisation egale au taux cible afin de minimiser les rachats conjoncturels.

\end{description}

\end{Description}
%
\begin{Usage}
\begin{verbatim}
besoin_revalo_epargne(epargne, cible)
\end{verbatim}
\end{Usage}
%
\begin{Arguments}
\begin{ldescription}
\item[\code{epargne}] est un objet de type \code{\LinkA{Epargne}{Epargne}}.

\item[\code{cible}] est un \code{numeric} representant un taux cible.
\end{ldescription}
\end{Arguments}
%
\begin{Author}\relax
Damien Tichit pour Sia Partners
\end{Author}
\inputencoding{utf8}
\HeaderA{besoin\_revalo\_ptf\_passif}{Fonction \code{besoin\_revalo\_ptf\_passif}}{besoin.Rul.revalo.Rul.ptf.Rul.passif}
%
\begin{Description}\relax
Cette fonction permet de calculer les besoins en revalorisation pour les differents PTF passifs d'une compagnie d'assurance :
\begin{description}

\item[besoins contractuels]  : besoins pour assouvir les engagements contractuels.
\item[besoins cibles]  : besoins pour atteindre une revalorisation cible.

\end{description}

\end{Description}
%
\begin{Usage}
\begin{verbatim}
besoin_revalo_ptf_passif(ptf_passif, cible)
\end{verbatim}
\end{Usage}
%
\begin{Arguments}
\begin{ldescription}
\item[\code{ptf\_passif}] est un objet de type \code{\LinkA{PTFPassif}{PTFPassif}}.

\item[\code{cible}] est une \code{list} contenant des elements relatifs a la politique de revalorisation pour les differents passifs.
\end{ldescription}
\end{Arguments}
%
\begin{Author}\relax
Damien Tichit pour Sia Partners
\end{Author}
\inputencoding{utf8}
\HeaderA{calc\_be}{Fonction \code{calc\_be}.}{calc.Rul.be}
%
\begin{Description}\relax
Cette fonction est une fonction centrale du package. Elle permet en effet de calculer un BEL.
\end{Description}
%
\begin{Usage}
\begin{verbatim}
calc_be(alm, parallel = FALSE, nb_core = 1L)
\end{verbatim}
\end{Usage}
%
\begin{Arguments}
\begin{ldescription}
\item[\code{alm}] est un objet de type \code{\LinkA{ALM}{ALM}} contenant l'ensemble des donnees.

\item[\code{parallel}] est une valeur \code{logical}. Lorsque cet argument est a \code{TRUE}, les calculs sont parallelises.

\item[\code{nb\_core}] est une valeur \code{integer} qui indique le nombre de coeurs utilises lorsque les calculs sont parallelises.
\end{ldescription}
\end{Arguments}
%
\begin{Details}\relax
Il est possible paralleliser les calculs afin d'accelerer le calcul d'un best-estimate.
\end{Details}
%
\begin{Author}\relax
Damien Tichit pour Sia Partners
\end{Author}
\inputencoding{utf8}
\HeaderA{calc\_be\_simu}{Fonction \code{calc\_be\_simu}.}{calc.Rul.be.Rul.simu}
%
\begin{Description}\relax
Cette fonction est une fonction centrale du package. Elle permet en effet de calculer un BEL pour une simulation donnee.
\end{Description}
%
\begin{Usage}
\begin{verbatim}
calc_be_simu(alm, num_sim)
\end{verbatim}
\end{Usage}
%
\begin{Arguments}
\begin{ldescription}
\item[\code{alm}] est un objet de type \code{ALM} contenant l'ensemble des donnees.

\item[\code{num\_sim}] est un \code{integer} representant le numero de simulation sur lequel on travaille.
\end{ldescription}
\end{Arguments}
%
\begin{Details}\relax
C'est sur cette fonction que s'effectue les boucles sur le nombre de simulations ainsi que sur les annees.
\end{Details}
%
\begin{Author}\relax
Damien Tichit pour Sia Partners
\end{Author}
%
\begin{SeeAlso}\relax
Projection sur une annee d'un \code{\LinkA{System}{System}} : \code{\LinkA{proj\_1an\_system}{proj.Rul.1an.Rul.system}}.
\end{SeeAlso}
\inputencoding{utf8}
\HeaderA{calc\_proba\_epargne}{Fonction \code{calc\_proba\_epargne}.}{calc.Rul.proba.Rul.epargne}
%
\begin{Description}\relax
Cette fonction permet de completer l'objet contenant les probabilites relatives au portefeuille epargne.
\end{Description}
%
\begin{Usage}
\begin{verbatim}
calc_proba_epargne(epargne, hyp_passif, an)
\end{verbatim}
\end{Usage}
%
\begin{Arguments}
\begin{ldescription}
\item[\code{epargne}] est un objet de type \code{\LinkA{Epargne}{Epargne}}.

\item[\code{hyp\_passif}] est un objet de type \code{\LinkA{HypPassif}{HypPassif}}.

\item[\code{an}] est un objet de type \code{integer}.
\end{ldescription}
\end{Arguments}
%
\begin{Author}\relax
Damien Tichit pour Sia Partners
\end{Author}
\inputencoding{utf8}
\HeaderA{calc\_proba\_ptf\_passif}{Fonction \code{calc\_proba\_ptf\_passif}}{calc.Rul.proba.Rul.ptf.Rul.passif}
%
\begin{Description}\relax
Cette fonction permet de calculer et de completer les objets relatifs aux probabilites des differents portefeuilles.
\end{Description}
%
\begin{Usage}
\begin{verbatim}
calc_proba_ptf_passif(ptf_passif, hyp_passif, an)
\end{verbatim}
\end{Usage}
%
\begin{Arguments}
\begin{ldescription}
\item[\code{ptf\_passif}] est un objet de type \code{\LinkA{PTFPassif}{PTFPassif}}.

\item[\code{hyp\_passif}] est un objet de type \code{\LinkA{HypPassif}{HypPassif}}.

\item[\code{an}] est un objet de type \code{integer}.
\end{ldescription}
\end{Arguments}
%
\begin{Author}\relax
Damien Tichit pour Sia Partners
\end{Author}
\inputencoding{utf8}
\HeaderA{calc\_qx}{Fonction \code{calc\_qx}.}{calc.Rul.qx}
%
\begin{Description}\relax
Cette fonction permet de calculer des probas de deces pour un age et une table de mortalite donnes.
Il est possible d'indiquer plusieurs ages differents sous forme de vecteur.
\end{Description}
%
\begin{Usage}
\begin{verbatim}
calc_qx(tab_morta, age)
\end{verbatim}
\end{Usage}
%
\begin{Arguments}
\begin{ldescription}
\item[\code{tab\_morta}] est un objet de type \code{\LinkA{TabMorta}{TabMorta}}.

\item[\code{age}] est un \code{integer}.
\end{ldescription}
\end{Arguments}
%
\begin{Author}\relax
Damien Tichit pour Sia Partners
\end{Author}
\inputencoding{utf8}
\HeaderA{calc\_rachat\_conj}{Fonction \code{calc\_rachat\_conj}.}{calc.Rul.rachat.Rul.conj}
%
\begin{Description}\relax
Cette fonction permet de calculer les taux de rachats conjoncturels.
\end{Description}
%
\begin{Usage}
\begin{verbatim}
calc_rachat_conj(rachat_conj, tx_cible, tx_serv)
\end{verbatim}
\end{Usage}
%
\begin{Arguments}
\begin{ldescription}
\item[\code{rachat\_conj}] est un objet de type \code{\LinkA{RachatConj}{RachatConj}}.

\item[\code{tx\_cible}] est un \code{numeric}.

\item[\code{tx\_serv}] est un \code{numeric}.
\end{ldescription}
\end{Arguments}
%
\begin{Author}\relax
Damien Tichit pour Sia Partners
\end{Author}
\inputencoding{utf8}
\HeaderA{calc\_rx}{Fonction \code{calc\_rx}.}{calc.Rul.rx}
%
\begin{Description}\relax
Cette fonction permet de calculer des probas de rachat pour une anciennete et une table de rachat donnes.
Il est possible d'indiquer plusieurs ages differents sous forme de vecteur.
\end{Description}
%
\begin{Usage}
\begin{verbatim}
calc_rx(tab_rachat, anc)
\end{verbatim}
\end{Usage}
%
\begin{Arguments}
\begin{ldescription}
\item[\code{tab\_rachat}] est un objet de type \code{\LinkA{TabRachat}{TabRachat}}.

\item[\code{anc}] est un \code{integer}.
\end{ldescription}
\end{Arguments}
%
\begin{Author}\relax
Damien Tichit pour Sia Partners
\end{Author}
\inputencoding{utf8}
\HeaderA{calc\_spread}{Fonction \code{calc\_spread}}{calc.Rul.spread}
%
\begin{Description}\relax
Cette fonction permet de calculer les spread pour un PTF obligataire.
\end{Description}
%
\begin{Usage}
\begin{verbatim}
calc_spread(obligation, yield_curve)
\end{verbatim}
\end{Usage}
%
\begin{Arguments}
\begin{ldescription}
\item[\code{obligation}] est un objet de type \code{\LinkA{Obligation}{Obligation}}.

\item[\code{yield\_curve}] est un vecteur \code{numeric}.
\end{ldescription}
\end{Arguments}
%
\begin{Author}\relax
Damien Tichit pour Sia Partners
\end{Author}
\inputencoding{utf8}
\HeaderA{calc\_tri}{Fonction \code{calc\_tri}}{calc.Rul.tri}
%
\begin{Description}\relax
Cette fonction permet de calculer les taux actuariels pour un PTF obligataire.
\end{Description}
%
\begin{Usage}
\begin{verbatim}
calc_tri(obligation)
\end{verbatim}
\end{Usage}
%
\begin{Arguments}
\begin{ldescription}
\item[\code{obligation}] est un objet de type \code{\LinkA{Obligation}{Obligation}}.
\end{ldescription}
\end{Arguments}
%
\begin{Author}\relax
Damien Tichit pour Sia Partners
\end{Author}
\inputencoding{utf8}
\HeaderA{calcul\_duration\_obligation}{Cette fonction permet de calculer les duration pour un portfeuille obligataire.}{calcul.Rul.duration.Rul.obligation}
%
\begin{Description}\relax
Cette fonction permet de calculer les duration pour un portfeuille obligataire.
\end{Description}
%
\begin{Arguments}
\begin{ldescription}
\item[\code{coupon}] un vecteur contenant \code{numeric} les coupons.

\item[\code{mat\_res}] un vecteur \code{numeric} contenant les maturites residuelles.

\item[\code{valeur\_remboursement}] un vecteur \code{numeric} contenant les valeurs de remboursement.

\item[\code{yield}] un vecteur contenant la courbe de taux utilisee.
\end{ldescription}
\end{Arguments}
%
\begin{Author}\relax
Damien Tichit pour Sia Partners
\end{Author}
\inputencoding{utf8}
\HeaderA{calcul\_fonds\_propres}{Fonction \code{calcul\_fonds\_propres}}{calcul.Rul.fonds.Rul.propres}
%
\begin{Description}\relax
Cette fonction permet de determiner le montant total des fonds propres.
\end{Description}
%
\begin{Usage}
\begin{verbatim}
calcul_fonds_propres(fp)
\end{verbatim}
\end{Usage}
%
\begin{Arguments}
\begin{ldescription}
\item[\code{fp}] est un objet de type \code{FondsPropres}.
\end{ldescription}
\end{Arguments}
%
\begin{Author}\relax
Damien Tichit pour Sia Partners
\end{Author}
\inputencoding{utf8}
\HeaderA{calcul\_pb}{Fonction \code{calcul\_pb}.}{calcul.Rul.pb}
%
\begin{Description}\relax
Cette fonction permet de determiner la PB a distribuer.
\end{Description}
%
\begin{Usage}
\begin{verbatim}
calcul_pb(taux_pb, resultat_fin, resultat_tech)
\end{verbatim}
\end{Usage}
%
\begin{Arguments}
\begin{ldescription}
\item[\code{taux\_pb}] est une \code{list} contenant les deux taux de pb contractuels.

\item[\code{resultat\_fin}] est un \code{numeric}

\item[\code{resultat\_tech}] est un \code{numeric}
\end{ldescription}
\end{Arguments}
%
\begin{Author}\relax
Damien Tichit pour Sia Partners
\end{Author}
\inputencoding{utf8}
\HeaderA{calcul\_pm}{Fonction \code{calcul\_pm}}{calcul.Rul.pm}
%
\begin{Description}\relax
Cette fonction permet de determiner le montant total des PM dans les differents portefeuilles de la compagnie d'assurance.
\end{Description}
%
\begin{Usage}
\begin{verbatim}
calcul_pm(ptf_passif)
\end{verbatim}
\end{Usage}
%
\begin{Arguments}
\begin{ldescription}
\item[\code{ptf\_passif}] est un objet de type \code{\LinkA{PTFPassif}{PTFPassif}}.
\end{ldescription}
\end{Arguments}
%
\begin{Author}\relax
Damien Tichit pour Sia Partners
\end{Author}
\inputencoding{utf8}
\HeaderA{calcul\_pt}{Fonction \code{calcul\_pt}}{calcul.Rul.pt}
%
\begin{Description}\relax
Cette fonction permet de determiner le montant total des provisions techniques pour une compagnie d'assurance : PM et PPE
\end{Description}
%
\begin{Usage}
\begin{verbatim}
calcul_pt(passif)
\end{verbatim}
\end{Usage}
%
\begin{Arguments}
\begin{ldescription}
\item[\code{passif}] est un objet de type \code{\LinkA{Passif}{Passif}}.
\end{ldescription}
\end{Arguments}
%
\begin{Author}\relax
Damien Tichit pour Sia Partners
\end{Author}
\inputencoding{utf8}
\HeaderA{calcul\_quote\_part\_fp}{Fonction \code{calcul\_quote\_part\_fp}}{calcul.Rul.quote.Rul.part.Rul.fp}
%
\begin{Description}\relax
Cette fonction permet de determiner la proportion que represente les fonds propres sur le passif d'une compagnie d'assurance.
\end{Description}
%
\begin{Usage}
\begin{verbatim}
calcul_quote_part_fp(passif)
\end{verbatim}
\end{Usage}
%
\begin{Arguments}
\begin{ldescription}
\item[\code{passif}] est un objet de type \code{\LinkA{Passif}{Passif}}.

\item[\code{result\_fin}] est un \code{numeric} representant le resultat financier total.
\end{ldescription}
\end{Arguments}
%
\begin{Author}\relax
Damien Tichit pour Sia Partners
\end{Author}
\inputencoding{utf8}
\HeaderA{calcul\_resultat}{Fonction \code{calcul\_resultat}.}{calcul.Rul.resultat}
%
\begin{Description}\relax
Cette fonction permet de calculer le resultat de l'exercice en cours pour une compagnie d'assurance.
\end{Description}
%
\begin{Usage}
\begin{verbatim}
calcul_resultat(resultat)
\end{verbatim}
\end{Usage}
%
\begin{Arguments}
\begin{ldescription}
\item[\code{resultat}] est une \code{list} contenant les differents elements du resultat.
\end{ldescription}
\end{Arguments}
%
\begin{Author}\relax
Damien Tichit pour Sia Partners
\end{Author}
\inputencoding{utf8}
\HeaderA{calcul\_resultat\_fin}{Fonction \code{calcul\_resultat\_fin}.}{calcul.Rul.resultat.Rul.fin}
%
\begin{Description}\relax
Cette fonction permet de calculer le resultat financier d'une compagnie d'assurance.
\end{Description}
%
\begin{Usage}
\begin{verbatim}
calcul_resultat_fin(resultat_fin)
\end{verbatim}
\end{Usage}
%
\begin{Arguments}
\begin{ldescription}
\item[\code{resultat\_fin}] est une \code{list} contenant les resultats financiers : PMVR, produits financiers et frais.
\end{ldescription}
\end{Arguments}
%
\begin{Author}\relax
Damien Tichit pour Sia Partners
\end{Author}
\inputencoding{utf8}
\HeaderA{calcul\_resultat\_tech}{Fonction \code{calcul\_resultat\_tech}.}{calcul.Rul.resultat.Rul.tech}
%
\begin{Description}\relax
Cette fonction permet de calculer le resultat technique d'une compagnie d'assurance.
\end{Description}
%
\begin{Usage}
\begin{verbatim}
calcul_resultat_tech(resultat_tech)
\end{verbatim}
\end{Usage}
%
\begin{Arguments}
\begin{ldescription}
\item[\code{resultat\_tech}] est une \code{list} contenant le resultat technique : chargements
\end{ldescription}
\end{Arguments}
%
\begin{Author}\relax
Damien Tichit pour Sia Partners
\end{Author}
\inputencoding{utf8}
\HeaderA{calcul\_revalo}{Fonction \code{calcul\_revalo}}{calcul.Rul.revalo}
%
\begin{Description}\relax
Cette fonction permet de calculer la revalorisation applique en fonction du besoin, de la PB ainsi que de la PPE.
\end{Description}
%
\begin{Usage}
\begin{verbatim}
calcul_revalo(besoin, pb, ppe, revalo_oblig = 0)
\end{verbatim}
\end{Usage}
%
\begin{Arguments}
\begin{ldescription}
\item[\code{besoin}] est un \code{numeric}.

\item[\code{pb}] est un \code{numeric} representant le montant de PB restant.

\item[\code{ppe}] est un objet de type \code{\LinkA{PPE}{PPE}}.

\item[\code{revalo\_oblig}] est un \code{numeric}. Il represente un montant de revalorisation obligatoire devant encore etre attribue.
\end{ldescription}
\end{Arguments}
%
\begin{Author}\relax
Damien Tichit pour Sia Partners
\end{Author}
\inputencoding{utf8}
\HeaderA{calcul\_vm\_obligation}{Cette fonction permet de calculer les valeurs de marche pour un portfeuille obligataire.}{calcul.Rul.vm.Rul.obligation}
%
\begin{Description}\relax
Cette fonction permet de calculer les valeurs de marche pour un portfeuille obligataire.
\end{Description}
%
\begin{Arguments}
\begin{ldescription}
\item[\code{coupon}] un vecteur contenant \code{numeric} les coupons.

\item[\code{mat\_res}] un vecteur \code{numeric} contenant les maturites residuelles.

\item[\code{valeur\_remboursement}] un vecteur \code{numeric} contenant les valeurs de remboursement.

\item[\code{spread}] un vecteur \code{numeric} contenant les spread.

\item[\code{yield}] un vecteur contenant la courbe de taux utilisee.
\end{ldescription}
\end{Arguments}
%
\begin{Author}\relax
Damien Tichit pour Sia Partners
\end{Author}
\inputencoding{utf8}
\HeaderA{dotation\_ppe}{Fonction \code{dotation\_ppe}.}{dotation.Rul.ppe}
%
\begin{Description}\relax
Cette fonction permet de doter la PPE. Le montant est dote sur la 1ere annee de reserve.

Cette fonction permet de reprendre ou doter un montant a la PPE.
\begin{description}

\item[dotation]  : Appel de la fonction \code{\LinkA{dotation\_ppe}{dotation.Rul.ppe}} ;
\item[reprise]  : Appel de la fonction \code{\LinkA{reprise\_ppe}{reprise.Rul.ppe}}.

\end{description}

\end{Description}
%
\begin{Usage}
\begin{verbatim}
dotation_ppe(ppe, montant)

dotation_reprise_ppe(ppe, montant)
\end{verbatim}
\end{Usage}
%
\begin{Arguments}
\begin{ldescription}
\item[\code{ppe}] est un objet de type \code{\LinkA{PPE}{PPE}}.

\item[\code{montant}] est un \code{numeric} representant le montant a doter.

\item[\code{ppe}] est un objet de type \code{\LinkA{PPE}{PPE}}.

\item[\code{montant}] est un \code{numeric} representant le montant a doter ou a reprendre.
\end{ldescription}
\end{Arguments}
%
\begin{Author}\relax
Damien Tichit pour Sia Partners

Damien Tichit pour Sia Partners
\end{Author}
\inputencoding{utf8}
\HeaderA{dotation\_reserve\_capi}{Fonction \code{dotation\_pre}.}{dotation.Rul.reserve.Rul.capi}
%
\begin{Description}\relax
Cette fonction permet de doter la PRE.

Cette fonction permet de doter la Reserve de capitalisation. Sa modelisation reflete sa definition (Article A333.3).
\end{Description}
%
\begin{Usage}
\begin{verbatim}
dotation_pre(pre, pmvl)

dotation_reserve_capi(reserve_capi, pmvr)
\end{verbatim}
\end{Usage}
%
\begin{Arguments}
\begin{ldescription}
\item[\code{pmvl}] est une \code{list} contenant les differentes PMVL par classes d'actifs modelisees.

\item[\code{reserve\_capi}] est un objet de type \code{\LinkA{PRE}{PRE}}.

\item[\code{pmvr}] est un \code{numeric} representant le montant les plus ou moins value resalisees a doter.

\item[\code{reserve\_capi}] est un objet de type \code{\LinkA{ReserveCapi}{ReserveCapi}}.
\end{ldescription}
\end{Arguments}
%
\begin{Details}\relax
En cas de plus value (montant > 0) : dotation totale.
En cas de moins value (montant < 0) : reprise dans la reserve de capitalisation, sous reserve qu'il soit encore positif.
\end{Details}
%
\begin{Author}\relax
Damien Tichit pour Sia Partners

Damien Tichit pour Sia Partners
\end{Author}
\inputencoding{utf8}
\HeaderA{Epargne}{Classe \code{Epargne}}{Epargne}
\keyword{classes}{Epargne}
%
\begin{Description}\relax
Cette classe represente le portefeuille des contrats epargne de la compagnie d'assurance.
\end{Description}
%
\begin{Section}{Slots}

\begin{description}

\item[\code{ptf}] est un objet de type \code{data.frame} contenant les donnees relatives au portfeuille.

\item[\code{proba}] est un objet de type \code{\LinkA{ProbaEpargne}{ProbaEpargne}} contenant les probas relatives au portfeuille.

\end{description}
\end{Section}
%
\begin{Author}\relax
Damien Tichit pour Sia Partners
\end{Author}
\inputencoding{utf8}
\HeaderA{ESG}{Classe \code{ESG}}{ESG}
\keyword{classes}{ESG}
%
\begin{Description}\relax
Cette classe aggrege l'ensemble des donnees relatives  a un ESG. Les donnees sont renseignees par simulation ainsi que par annee de projection.
\end{Description}
%
\begin{Section}{Slots}

\begin{description}

\item[\code{coef\_actu}] est un \code{data.table} : coefficients d'actualisation.

\item[\code{ctz\_nom}] est un \code{data.table} : taux zero coupon reels.

\item[\code{ctz\_reel}] est un \code{data.table} : taux zero coupon nominaux.

\item[\code{eq\_dividends}] est un \code{data.table} : dividendes verses.

\item[\code{eq\_index}] est un \code{data.table} : log-rendements du PTF action.

\item[\code{im\_index}] est un \code{data.table} : log-rendements du PTF immobilier

\item[\code{im\_loyer}] est un \code{data.table} : loyers verses.

\item[\code{inflation}] est un \code{data.table} : taux d'inflation.

\end{description}
\end{Section}
%
\begin{Author}\relax
Damien Tichit pour Sia Partners
\end{Author}
\inputencoding{utf8}
\HeaderA{eval\_frais\_fin}{Fonction \code{eval\_frais\_fin}}{eval.Rul.frais.Rul.fin}
%
\begin{Description}\relax
Cette fonction permet d'evaluer les frais relatifs au portfeuille actif d'une compagnie d'assurance. Il est possible de modeliser des frais par produits financiers
mais egalement des frais proportionnels aux VM.
\end{Description}
%
\begin{Usage}
\begin{verbatim}
eval_frais_fin(ptf_actif, frais_fin, prod_fin)
\end{verbatim}
\end{Usage}
%
\begin{Arguments}
\begin{ldescription}
\item[\code{ptf\_actif}] est un objet de type \code{\LinkA{PTFActif}{PTFActif}}.

\item[\code{frais\_fin}] est un objet de type \code{data.frame}.

\item[\code{prod\_fin}] est une \code{list} contenant les produits financiers pour les differentes classes d'actifs modelisees..
\end{ldescription}
\end{Arguments}
%
\begin{Author}\relax
Damien Tichit pour Sia Partners
\end{Author}
\inputencoding{utf8}
\HeaderA{extract\_pmvl\_action}{Fonction \code{extract\_pmvl\_action}}{extract.Rul.pmvl.Rul.action}
%
\begin{Description}\relax
Cette fonction permet de determiner les PMVL pouvant etre realises sur un PTF d'actions.
\end{Description}
%
\begin{Usage}
\begin{verbatim}
extract_pmvl_action(action)
\end{verbatim}
\end{Usage}
%
\begin{Arguments}
\begin{ldescription}
\item[\code{action}] est un objet de type \code{Action}.
\end{ldescription}
\end{Arguments}
%
\begin{Author}\relax
Damien Tichit pour Sia Partners
\end{Author}
\inputencoding{utf8}
\HeaderA{extract\_pmvl\_immobilier}{Fonction \code{extract\_pmvl\_immobilier}}{extract.Rul.pmvl.Rul.immobilier}
%
\begin{Description}\relax
Cette fonction permet de determiner les PMVL pouvant etre realises sur un PTF immobilier.
\end{Description}
%
\begin{Usage}
\begin{verbatim}
extract_pmvl_immobilier(immobilier)
\end{verbatim}
\end{Usage}
%
\begin{Arguments}
\begin{ldescription}
\item[\code{immobilier}] est un objet de type \code{Immobilier}.
\end{ldescription}
\end{Arguments}
%
\begin{Author}\relax
Damien Tichit pour Sia Partners
\end{Author}
\inputencoding{utf8}
\HeaderA{extract\_pmvl\_obligation}{Fonction \code{extract\_pmvl\_obligation}}{extract.Rul.pmvl.Rul.obligation}
%
\begin{Description}\relax
Cette fonction permet de determiner les PMVL pouvant etre realises sur un PTF obligataire.
\end{Description}
%
\begin{Usage}
\begin{verbatim}
extract_pmvl_obligation(obligation)
\end{verbatim}
\end{Usage}
%
\begin{Arguments}
\begin{ldescription}
\item[\code{obligation}] est un objet de type \code{Obligation}.
\end{ldescription}
\end{Arguments}
%
\begin{Author}\relax
Damien Tichit pour Sia Partners
\end{Author}
\inputencoding{utf8}
\HeaderA{extract\_pmvl\_ptf\_actif}{Fonction \code{extract\_pmvl\_ptf\_actif}}{extract.Rul.pmvl.Rul.ptf.Rul.actif}
%
\begin{Description}\relax
Cette fonction permet de determiner les PMVL pouvant etre realises sur les differents portefeuille.
\end{Description}
%
\begin{Usage}
\begin{verbatim}
extract_pmvl_ptf_actif(ptf_actif)
\end{verbatim}
\end{Usage}
%
\begin{Arguments}
\begin{ldescription}
\item[\code{ptf\_actif}] est un objet de type \code{PTFActif}.
\end{ldescription}
\end{Arguments}
%
\begin{Author}\relax
Damien Tichit pour Sia Partners
\end{Author}
\inputencoding{utf8}
\HeaderA{FondsPropres}{Classe \code{FondsPropres}}{FondsPropres}
\keyword{classes}{FondsPropres}
%
\begin{Description}\relax
Cette classe aggrege l'ensemble des donnees relatives au passif de la compagnie d'assurance : hypotheses, portefeuille, provisions
\end{Description}
%
\begin{Section}{Slots}

\begin{description}

\item[\code{capitaux\_propres}] est un \code{numeric} representant le montant des capitaux propres

\item[\code{report\_a\_nouveau}] est un \code{numeric} representant le montant du report a nouveau.

\item[\code{resultat\_exercice}] est un \code{numeric} representant le resultat de l'exercice en cours.

\item[\code{dette}] est un \code{numeric}.

\item[\code{hypotheses}] est une \code{list}.

\end{description}
\end{Section}
%
\begin{Author}\relax
Damien Tichit pour Sia Partners
\end{Author}
\inputencoding{utf8}
\HeaderA{gestion\_action}{Fonction \code{gestion\_action}}{gestion.Rul.action}
%
\begin{Description}\relax
Cette fonction permet de gerer un portfeuille action : recolte des dividendes, recalcul des VM
\end{Description}
%
\begin{Usage}
\begin{verbatim}
gestion_action(action, hyp_actif, an)
\end{verbatim}
\end{Usage}
%
\begin{Arguments}
\begin{ldescription}
\item[\code{action}] est un objet de type \code{\LinkA{Action}{Action}}.

\item[\code{hyp\_actif}] est un objet de type \code{\LinkA{HypActif}{HypActif}}.

\item[\code{an}] est un \code{integer} reprensentant l'annee sur laquelle on travaille.
\end{ldescription}
\end{Arguments}
%
\begin{Author}\relax
Damien Tichit pour Sia Partners
\end{Author}
\inputencoding{utf8}
\HeaderA{gestion\_capitaux\_propres}{Fonction \code{gestion\_capitaux\_propres}}{gestion.Rul.capitaux.Rul.propres}
%
\begin{Description}\relax
Cette fonction permet de gerer les capitaux propres : ajout d'une part du resultat fin, retirer les dividendes.
\end{Description}
%
\begin{Usage}
\begin{verbatim}
gestion_capitaux_propres(passif, result_fin)
\end{verbatim}
\end{Usage}
%
\begin{Arguments}
\begin{ldescription}
\item[\code{passif}] est un objet de type \code{\LinkA{Passif}{Passif}}.

\item[\code{result\_fin}] est un \code{numeric} representant le resultat financier total.
\end{ldescription}
\end{Arguments}
%
\begin{Author}\relax
Damien Tichit pour Sia Partners
\end{Author}
\inputencoding{utf8}
\HeaderA{gestion\_fin\_projection\_actif}{Fonction \code{gestion\_fin\_projection\_actif}}{gestion.Rul.fin.Rul.projection.Rul.actif}
%
\begin{Description}\relax
Cette fonction permet de gerer la fin de projection du modele sur l'actif de la compagnie d'assurance.
\end{Description}
%
\begin{Usage}
\begin{verbatim}
gestion_fin_projection_actif(actif)
\end{verbatim}
\end{Usage}
%
\begin{Arguments}
\begin{ldescription}
\item[\code{actif}] est un objet de type \code{\LinkA{Actif}{Actif}}.
\end{ldescription}
\end{Arguments}
%
\begin{Author}\relax
Damien Tichit pour Sia Partners
\end{Author}
\inputencoding{utf8}
\HeaderA{gestion\_fin\_projection\_passif}{Fonction \code{gestion\_fin\_projection\_passif}}{gestion.Rul.fin.Rul.projection.Rul.passif}
%
\begin{Description}\relax
Cette fonction permet de gerer la fin de projection du modele sur le passif de la compagnie d'assurance.
\end{Description}
%
\begin{Usage}
\begin{verbatim}
gestion_fin_projection_passif(passif)
\end{verbatim}
\end{Usage}
%
\begin{Arguments}
\begin{ldescription}
\item[\code{passif}] est un objet de type \code{\LinkA{Passif}{Passif}}.
\end{ldescription}
\end{Arguments}
%
\begin{Author}\relax
Damien Tichit pour Sia Partners
\end{Author}
\inputencoding{utf8}
\HeaderA{gestion\_fin\_projection\_system}{Fonction \code{gestion\_fin\_projection\_system}}{gestion.Rul.fin.Rul.projection.Rul.system}
%
\begin{Description}\relax
Cette fonction permet de gerer la fin de projection de la compagnie d'assurance.
\end{Description}
%
\begin{Usage}
\begin{verbatim}
gestion_fin_projection_system(system)
\end{verbatim}
\end{Usage}
%
\begin{Arguments}
\begin{ldescription}
\item[\code{system}] est un objet de type \code{\LinkA{System}{System}}.
\end{ldescription}
\end{Arguments}
%
\begin{Author}\relax
Damien Tichit pour Sia Partners
\end{Author}
\inputencoding{utf8}
\HeaderA{gestion\_fonds\_propres}{Fonction \code{gestion\_fonds\_propres}}{gestion.Rul.fonds.Rul.propres}
%
\begin{Description}\relax
Cette fonction permet de gerer les fonds propres.
\end{Description}
%
\begin{Usage}
\begin{verbatim}
gestion_fonds_propres(fp, resultat, emprunt)
\end{verbatim}
\end{Usage}
%
\begin{Arguments}
\begin{ldescription}
\item[\code{fp}] est un objet de type \code{FondsPropres}.

\item[\code{resultat}] est un \code{numeric} representant le resultat de l'annee.

\item[\code{emprunt}] est un \code{numeric}.
\end{ldescription}
\end{Arguments}
%
\begin{Author}\relax
Damien Tichit pour Sia Partners
\end{Author}
\inputencoding{utf8}
\HeaderA{gestion\_immobilier}{Fonction \code{gestion\_immobilier}}{gestion.Rul.immobilier}
%
\begin{Description}\relax
Cette fonction permet de gerer un portfeuille immobilier : recolte des loyers, recalcul des VM
\end{Description}
%
\begin{Usage}
\begin{verbatim}
gestion_immobilier(immobilier, hyp_actif, an)
\end{verbatim}
\end{Usage}
%
\begin{Arguments}
\begin{ldescription}
\item[\code{immobilier}] est un objet de type \code{\LinkA{Immobilier}{Immobilier}}.

\item[\code{hyp\_actif}] est un objet de type \code{\LinkA{HypActif}{HypActif}}.

\item[\code{an}] est un \code{integer} reprensentant l'annee sur laquelle on travaille.
\end{ldescription}
\end{Arguments}
%
\begin{Author}\relax
Damien Tichit pour Sia Partners
\end{Author}
\inputencoding{utf8}
\HeaderA{gestion\_obligation}{Fonction \code{gestion\_obligation}}{gestion.Rul.obligation}
%
\begin{Description}\relax
Cette fonction permet de gerer un portfeuille obligation : recolte des coupons, recalcul des VM, VNC, VR, vieillissement...
\end{Description}
%
\begin{Usage}
\begin{verbatim}
gestion_obligation(obligation, hyp_actif, an)
\end{verbatim}
\end{Usage}
%
\begin{Arguments}
\begin{ldescription}
\item[\code{obligation}] est un objet de type \code{\LinkA{Obligation}{Obligation}}.

\item[\code{hyp\_actif}] est un objet de type \code{\LinkA{HypActif}{HypActif}}.

\item[\code{an}] est un \code{integer} reprensentant l'annee sur laquelle on travaille.
\end{ldescription}
\end{Arguments}
%
\begin{Author}\relax
Damien Tichit pour Sia Partners
\end{Author}
\inputencoding{utf8}
\HeaderA{gestion\_ptf\_actif}{Fonction \code{gestion\_ptf\_actif}.}{gestion.Rul.ptf.Rul.actif}
%
\begin{Description}\relax
Cette fonction permet de gerer les differents ptf actif d'une compagnie d'assurance : recolte des produits financiers, recalcul des VM, vieillissement...
\end{Description}
%
\begin{Usage}
\begin{verbatim}
gestion_ptf_actif(ptf_actif, hyp_actif, an)
\end{verbatim}
\end{Usage}
%
\begin{Arguments}
\begin{ldescription}
\item[\code{ptf\_actif}] est un objet de type \code{\LinkA{PTFActif}{PTFActif}}.

\item[\code{hyp\_actif}] est un objet de type \code{\LinkA{HypActif}{HypActif}}.

\item[\code{an}] est un \code{integer} reprensentant l'annee sur laquelle on travaille.
\end{ldescription}
\end{Arguments}
%
\begin{Author}\relax
Damien Tichit pour Sia Partners
\end{Author}
\inputencoding{utf8}
\HeaderA{gestion\_tresorerie}{Fonction \code{gestion\_tresorerie}}{gestion.Rul.tresorerie}
%
\begin{Description}\relax
Cette fonction permet de gerer un portfeuille monetaire : recolte des interets monetaires.
\end{Description}
%
\begin{Usage}
\begin{verbatim}
gestion_tresorerie(tresorerie, hyp_actif, an)
\end{verbatim}
\end{Usage}
%
\begin{Arguments}
\begin{ldescription}
\item[\code{tresorerie}] est un objet de type \code{\LinkA{Tresorerie}{Tresorerie}}.

\item[\code{hyp\_actif}] est un objet de type \code{\LinkA{HypActif}{HypActif}}.

\item[\code{an}] est un \code{integer} reprensentant l'annee sur laquelle on travaille.
\end{ldescription}
\end{Arguments}
%
\begin{Author}\relax
Damien Tichit pour Sia Partners
\end{Author}
\inputencoding{utf8}
\HeaderA{HypActif}{Classe \code{HypActif}}{HypActif}
\keyword{classes}{HypActif}
%
\begin{Description}\relax
Cette classe aggrege l'ensemble des hypotheses relatives au passif de la compagnie d'assurance : portefeuille cible, frais financiers.
\end{Description}
%
\begin{Section}{Slots}

\begin{description}

\item[\code{ptf\_cible}] est un objet de la classe \code{\LinkA{PTFCible}{PTFCible}} representant le portfeuille financier cible.

\item[\code{esg\_simu}] est une \code{list} reprenant les donnees de l'ESG pour une simulation.

\item[\code{frais\_fin}] est un objet de la classe \code{\LinkA{data.frame}{data.frame}} contenant les hyptheses relatives aux frais financiers.

\item[\code{revalo\_treso}] est un \code{numeric} indiquant le taux auquel est revalorise la tresorerie.

\end{description}
\end{Section}
%
\begin{Author}\relax
Damien Tichit pour Sia Partners
\end{Author}
%
\begin{SeeAlso}\relax
Le parametre \code{esg\_simu} se met a jour grace a la fonction \code{\LinkA{update\_esg}{update.Rul.esg}}
\end{SeeAlso}
\inputencoding{utf8}
\HeaderA{HypALM}{Classe \code{HypALM}}{HypALM}
\keyword{classes}{HypALM}
%
\begin{Description}\relax
Cette classe aggrege l'ensemble des hypotheses relatives au modele ALM.
\end{Description}
%
\begin{Section}{Slots}

\begin{description}

\item[\code{esg}] est un objet de type \code{\LinkA{ESG}{ESG}}.

\item[\code{nb\_simu}] est un \code{integer} representant le nombre de simulation souhaite pour calculer le BEL.

\item[\code{an\_proj}] est un \code{integer} representant le nombre d'annees de projection.

\end{description}
\end{Section}
%
\begin{Author}\relax
Damien Tichit pour Sia Partners
\end{Author}
\inputencoding{utf8}
\HeaderA{HypPassif}{Classe \code{HypPassif}}{HypPassif}
\keyword{classes}{HypPassif}
%
\begin{Description}\relax
Cette classe aggrege l'ensemble des hypotheses relatives au passif d'une compagnie d'assurance : tables de rachat, de mortalite...
\end{Description}
%
\begin{Section}{Slots}

\begin{description}

\item[\code{tab\_morta\_h}] est un objet de la classe \code{\LinkA{TabMorta}{TabMorta}} contenant la table de mortalite pour les hommes.

\item[\code{tab\_morta\_f}] est un objet de la classe \code{\LinkA{TabMorta}{TabMorta}} contenant la table de mortalite pour les hommes.

\item[\code{tab\_rachat\_tot}] est un objet de la classe \code{\LinkA{TabRachat}{TabRachat}} contenant la table modelisant les rachats totaux.

\item[\code{tab\_rachat\_part}] est un objet de la classe \code{\LinkA{TabRachat}{TabRachat}} contenant la table modelisant les rachats partiels.

\item[\code{rachat\_conj}] est un objet de la classe \code{\LinkA{RachatConj}{RachatConj}} contenant les parametres modelisant les rachats conjoncturels.

\item[\code{calc\_proba}] est un objet \code{logical} qui indique si les probabilites doivent etre calculees.

\item[\code{prop\_pb}] est un \code{data.frame} qui indique la proportion a attribue dans le cas ou les besoins ont ete assouvis (apres besoins contractuels et cible).

\item[\code{cible}] est une \code{list} contenant les differents taux cibles par produits.

\item[\code{esg\_simu}] est une \code{list} reprenant les donnees de l'ESG pour une simulation : taux cibles et inflation.

\item[\code{dividende}] est un \code{numeric} representant le taux de dividendes verse chaque annee aux actionnaires.

\end{description}
\end{Section}
%
\begin{Author}\relax
Damien Tichit pour Sia Partners
\end{Author}
%
\begin{SeeAlso}\relax
La mise a jour de l'attribut \code{cible} : \code{\LinkA{update\_esg}{update.Rul.esg}}
\end{SeeAlso}
\inputencoding{utf8}
\HeaderA{Immobilier}{Classe \code{Immobilier}}{Immobilier}
\keyword{classes}{Immobilier}
%
\begin{Description}\relax
Cette classe represente le portefeuille des immobilier de la compagnie d'assurance.
\end{Description}
%
\begin{Section}{Slots}

\begin{description}

\item[\code{ptf}] est un objet de type \code{\LinkA{data.frame}{data.frame}} contenant les donnees relatives au portefeuille.

\end{description}
\end{Section}
%
\begin{Author}\relax
Damien Tichit pour Sia Partners
\end{Author}
\inputencoding{utf8}
\HeaderA{initialisation\_alm}{Fonction \code{initialisation\_alm}.}{initialisation.Rul.alm}
%
\begin{Description}\relax
Cette fonction permet d'initialiser un objet de type \code{\LinkA{ALM}{ALM}} : chargement des donnees, aggregation, calcul des probas...
\end{Description}
%
\begin{Usage}
\begin{verbatim}
initialisation_alm(alm)
\end{verbatim}
\end{Usage}
%
\begin{Arguments}
\begin{ldescription}
\item[\code{alm}] est un objet de type \code{\LinkA{ALM}{ALM}}.
\end{ldescription}
\end{Arguments}
%
\begin{Author}\relax
Damien Tichit pour Sia Partners
\end{Author}
%
\begin{SeeAlso}\relax
Construction d'un objet de type \code{\LinkA{ALM}{ALM}} : \code{\LinkA{load\_alm}{load.Rul.alm}}.

Premiere aggregation d'un objet de type \code{\LinkA{ALM}{ALM}} : \code{\LinkA{aggregation\_alm}{aggregation.Rul.alm}}.

Seconde aggregation pour les passifs \code{\LinkA{ALM}{ALM}} : \code{\LinkA{aggregation\_passif\_2}{aggregation.Rul.passif.Rul.2}}.
\end{SeeAlso}
\inputencoding{utf8}
\HeaderA{load\_actif}{Fonction \code{load\_actif}.}{load.Rul.actif}
%
\begin{Description}\relax
Cette fonction permet de charger les donnees pour un objet de type \code{\LinkA{Actif}{Actif}}. Les donnees auront ete prealablement disposees dans
une architecture propre a \code{SiALM}.
\end{Description}
%
\begin{Usage}
\begin{verbatim}
load_actif(address)
\end{verbatim}
\end{Usage}
%
\begin{Arguments}
\begin{ldescription}
\item[\code{address}] est un objet de type \code{character} indiquant le dossier dans lequel se situe l'ensemble des donnees necessaires
pour la construction de l'objet.
\end{ldescription}
\end{Arguments}
%
\begin{Details}\relax
La creation d'un objet \code{\LinkA{Actif}{Actif}} necessite la creation de deux autres objets : \code{\LinkA{PTFActif}{PTFActif}}, \code{\LinkA{HypActif}{HypActif}}.
\end{Details}
%
\begin{Author}\relax
Damien Tichit pour Sia Partners
\end{Author}
%
\begin{SeeAlso}\relax
Construction d'un objet de type \code{\LinkA{PTFActif}{PTFActif}} : \code{\LinkA{load\_ptf\_actif}{load.Rul.ptf.Rul.actif}}.

Construction d'un objet de type \code{\LinkA{HypActif}{HypActif}} : \code{\LinkA{load\_hyp\_actif}{load.Rul.hyp.Rul.actif}}.
\end{SeeAlso}
\inputencoding{utf8}
\HeaderA{load\_action}{Fonction \code{load\_action}.}{load.Rul.action}
%
\begin{Description}\relax
Cette fonction permet de charger les donnees pour un objet de type \code{\LinkA{Action}{Action}}. Les donnees auront ete prealablement disposees dans
une architecture propre a \code{SiALM}.
\end{Description}
%
\begin{Usage}
\begin{verbatim}
load_action(address)
\end{verbatim}
\end{Usage}
%
\begin{Arguments}
\begin{ldescription}
\item[\code{address}] est un objet de type \code{character} indiquant le dossier dans lequel se situe l'ensemble des donnees necessaires
pour la construction de l'objet.
\end{ldescription}
\end{Arguments}
%
\begin{Details}\relax
La creation d'un objet \code{\LinkA{Action}{Action}} necessite des donnees presentes dans un fichier nomme \code{Actions.csv}.
\end{Details}
%
\begin{Author}\relax
Damien Tichit pour Sia Partners
\end{Author}
\inputencoding{utf8}
\HeaderA{load\_alm}{Fonction \code{load\_alm}.}{load.Rul.alm}
%
\begin{Description}\relax
Cette fonction permet de charger les donnees pour un objet de type \code{\LinkA{ALM}{ALM}}. Les donnees auront ete prealablement disposees dans
une architecture propre a \code{SiALM}.
\end{Description}
%
\begin{Usage}
\begin{verbatim}
load_alm(address)
\end{verbatim}
\end{Usage}
%
\begin{Arguments}
\begin{ldescription}
\item[\code{address}] est un objet de type \code{character} indiquant le dossier dans lequel se situe l'ensemble des donnees necessaire
pour la construction de l'objet.
\end{ldescription}
\end{Arguments}
%
\begin{Details}\relax
La creation d'un objet \code{\LinkA{ALM}{ALM}} necessite la creation de deux autres objets : \code{\LinkA{System}{System}} et \code{\LinkA{HypALM}{HypALM}}.
\end{Details}
%
\begin{Author}\relax
Damien Tichit pour Sia Partners
\end{Author}
%
\begin{SeeAlso}\relax
Construction d'un objet de type \code{\LinkA{System}{System}} : \code{\LinkA{load\_system}{load.Rul.system}}.

Construction d'un objet de type \code{\LinkA{HypALM}{HypALM}} : \code{\LinkA{load\_hyp\_alm}{load.Rul.hyp.Rul.alm}}.
\end{SeeAlso}
\inputencoding{utf8}
\HeaderA{load\_epargne}{Fonction \code{load\_epargne}.}{load.Rul.epargne}
%
\begin{Description}\relax
Cette fonction permet de charger les donnees pour un objet de type \code{\LinkA{Epargne}{Epargne}}. Les donnees auront ete prealablement disposees dans
une architecture propre a \code{SiALM}.
\end{Description}
%
\begin{Usage}
\begin{verbatim}
load_epargne(address)
\end{verbatim}
\end{Usage}
%
\begin{Arguments}
\begin{ldescription}
\item[\code{address}] est un objet de type \code{character} indiquant le dossier dans lequel se situe l'ensemble des donnees necessaires
pour la construction de l'objet.
\end{ldescription}
\end{Arguments}
%
\begin{Details}\relax
La creation d'un objet \code{\LinkA{Epargne}{Epargne}} necessite des donnees presentes dans un fichier nomme \code{Epargne.csv}.
\end{Details}
%
\begin{Author}\relax
Damien Tichit pour Sia Partners
\end{Author}
\inputencoding{utf8}
\HeaderA{load\_esg}{Fonction \code{load\_esg}.}{load.Rul.esg}
%
\begin{Description}\relax
Cette fonction permet de charger les donnees pour un objet de type \code{\LinkA{ESG}{ESG}}. Les donnees auront ete prealablement disposees dans
une architecture propre a \code{SiALM}.
\end{Description}
%
\begin{Usage}
\begin{verbatim}
load_esg(address, avec_va = TRUE)
\end{verbatim}
\end{Usage}
%
\begin{Arguments}
\begin{ldescription}
\item[\code{address}] est un objet de type \code{character} indiquant le dossier dans lequel se situe l'ensemble des donnees necessaires
pour la construction de l'objet.

\item[\code{avec\_va}] est un \code{logical} indiquant si les donnees doivent etre chargees sur le jeu de donnees avec Volatility Adjustment. Par defaut, la valeur est a \code{TRUE}.
\end{ldescription}
\end{Arguments}
%
\begin{Author}\relax
Damien Tichit pour Sia Partners
\end{Author}
\inputencoding{utf8}
\HeaderA{load\_fonds\_propres}{Fonction \code{load\_fonds\_propres}.}{load.Rul.fonds.Rul.propres}
%
\begin{Description}\relax
Cette fonction permet de charger les donnees pour un objet de type \code{\LinkA{FondsPropres}{FondsPropres}}. Les donnees auront ete prealablement disposees dans
une architecture propre a \code{SiALM}.
\end{Description}
%
\begin{Usage}
\begin{verbatim}
load_fonds_propres(address)
\end{verbatim}
\end{Usage}
%
\begin{Arguments}
\begin{ldescription}
\item[\code{address}] est un objet de type \code{character} indiquant le dossier dans lequel se situe l'ensemble des donnees necessaires
pour la construction de l'objet.
\end{ldescription}
\end{Arguments}
%
\begin{Author}\relax
Damien Tichit pour Sia Partners
\end{Author}
\inputencoding{utf8}
\HeaderA{load\_hyp\_actif}{Fonction \code{load\_hyp\_actif}.}{load.Rul.hyp.Rul.actif}
%
\begin{Description}\relax
Cette fonction permet de charger les donnees pour un objet de type \code{\LinkA{HypActif}{HypActif}}. Les donnees auront ete prealablement disposees dans
une architecture propre a \code{SiALM}.
\end{Description}
%
\begin{Usage}
\begin{verbatim}
load_hyp_actif(address)
\end{verbatim}
\end{Usage}
%
\begin{Arguments}
\begin{ldescription}
\item[\code{address}] est un objet de type \code{character} indiquant le dossier dans lequel se situe l'ensemble des donnees necessaires
pour la construction de l'objet.
\end{ldescription}
\end{Arguments}
%
\begin{Details}\relax
La creation d'un objet \code{\LinkA{HypActif}{HypActif}} necessite la creation d'un autre objet : \code{\LinkA{PTFCible}{PTFCible}}.
\end{Details}
%
\begin{Author}\relax
Damien Tichit pour Sia Partners
\end{Author}
%
\begin{SeeAlso}\relax
Construction d'un objet de type \code{\LinkA{PTFCible}{PTFCible}} : \code{\LinkA{load\_ptf\_cible}{load.Rul.ptf.Rul.cible}}.
\end{SeeAlso}
\inputencoding{utf8}
\HeaderA{load\_hyp\_alm}{Fonction \code{load\_hyp\_alm}.}{load.Rul.hyp.Rul.alm}
%
\begin{Description}\relax
Cette fonction permet de charger les donnees pour un objet de type \code{\LinkA{HypALM}{HypALM}}. Les donnees auront ete prealablement disposees dans
une architecture propre a \code{SiALM}.
\end{Description}
%
\begin{Usage}
\begin{verbatim}
load_hyp_alm(address)
\end{verbatim}
\end{Usage}
%
\begin{Arguments}
\begin{ldescription}
\item[\code{address}] est un objet de type \code{character} indiquant le dossier dans lequel se situe l'ensemble des donnees necessaires
pour la construction de l'objet.
\end{ldescription}
\end{Arguments}
%
\begin{Details}\relax
La creation d'un objet \code{\LinkA{HypALM}{HypALM}} necessite plusieurs parametres presents dans un fichier nomme \code{Hypotheses\_ALM.csv}.
\end{Details}
%
\begin{Author}\relax
Damien Tichit pour Sia Partners
\end{Author}
%
\begin{SeeAlso}\relax
Chargement d'un objet \code{\LinkA{ESG}{ESG}} : \code{\LinkA{load\_esg}{load.Rul.esg}}
\end{SeeAlso}
\inputencoding{utf8}
\HeaderA{load\_hyp\_passif}{Fonction \code{load\_hyp\_passif}.}{load.Rul.hyp.Rul.passif}
%
\begin{Description}\relax
Cette fonction permet de charger les donnees pour un objet de type \code{\LinkA{HypPassif}{HypPassif}}. Les donnees auront ete prealablement disposees dans
une architecture propre a \code{SiALM}.
\end{Description}
%
\begin{Usage}
\begin{verbatim}
load_hyp_passif(address)
\end{verbatim}
\end{Usage}
%
\begin{Arguments}
\begin{ldescription}
\item[\code{address}] est un objet de type \code{character} indiquant le dossier dans lequel se situe l'ensemble des donnees necessaires
pour la construction de l'objet.
\end{ldescription}
\end{Arguments}
%
\begin{Details}\relax
La creation d'un objet \code{\LinkA{HypPassif}{HypPassif}} necessite la creation de deux autres objets differents : \code{\LinkA{TabMorta}{TabMorta}} et \code{\LinkA{TabRachat}{TabRachat}}.
\end{Details}
%
\begin{Author}\relax
Damien Tichit pour Sia Partners
\end{Author}
%
\begin{SeeAlso}\relax
Construction d'un objet de type \code{\LinkA{TabMorta}{TabMorta}} : \code{\LinkA{load\_tab\_morta}{load.Rul.tab.Rul.morta}}.

Construction d'un objet de type \code{\LinkA{TabRachat}{TabRachat}} : \code{\LinkA{load\_tab\_rachat}{load.Rul.tab.Rul.rachat}}.
\end{SeeAlso}
\inputencoding{utf8}
\HeaderA{load\_immobilier}{Fonction \code{load\_immobilier}.}{load.Rul.immobilier}
%
\begin{Description}\relax
Cette fonction permet de charger les donnees pour un objet de type \code{\LinkA{Immobilier}{Immobilier}}. Les donnees auront ete prealablement disposees dans
une architecture propre a \code{SiALM}.
\end{Description}
%
\begin{Usage}
\begin{verbatim}
load_immobilier(address)
\end{verbatim}
\end{Usage}
%
\begin{Arguments}
\begin{ldescription}
\item[\code{address}] est un objet de type \code{character} indiquant le dossier dans lequel se situe l'ensemble des donnees necessaires
pour la construction de l'objet.
\end{ldescription}
\end{Arguments}
%
\begin{Details}\relax
La creation d'un objet \code{\LinkA{Immobilier}{Immobilier}} necessite des donnees presentes dans un fichier nomme \code{Immobilier.csv}.
\end{Details}
%
\begin{Author}\relax
Damien Tichit pour Sia Partners
\end{Author}
\inputencoding{utf8}
\HeaderA{load\_obligation}{Fonction \code{load\_obligation}.}{load.Rul.obligation}
%
\begin{Description}\relax
Cette fonction permet de charger les donnees pour un objet de type \code{\LinkA{Obligation}{Obligation}}. Les donnees auront ete prealablement disposees dans
une architecture propre a \code{SiALM}.
\end{Description}
%
\begin{Usage}
\begin{verbatim}
load_obligation(address)
\end{verbatim}
\end{Usage}
%
\begin{Arguments}
\begin{ldescription}
\item[\code{address}] est un objet de type \code{character} indiquant le dossier dans lequel se situe l'ensemble des donnees necessaires
pour la construction de l'objet.
\end{ldescription}
\end{Arguments}
%
\begin{Details}\relax
La creation d'un objet \code{\LinkA{Obligation}{Obligation}} necessite des donnees presentes dans un fichier nomme \code{Obligations.csv}.

Cette fonction permet egalement d'aggreger les donnees : 1 obligation par maturite residuelle.
\end{Details}
%
\begin{Author}\relax
Damien Tichit pour Sia Partners
\end{Author}
\inputencoding{utf8}
\HeaderA{load\_obligation\_cible}{Fonction \code{load\_obligation\_cible}.}{load.Rul.obligation.Rul.cible}
%
\begin{Description}\relax
Cette fonction permet de charger les donnees pour un objet de type \code{\LinkA{Obligation}{Obligation}}. Les donnees auront ete prealablement disposees dans
une architecture propre a \code{SiALM}. Cette fonction sera utilisee pour le chargement du portfeuille cible.
\end{Description}
%
\begin{Usage}
\begin{verbatim}
load_obligation_cible(address)
\end{verbatim}
\end{Usage}
%
\begin{Arguments}
\begin{ldescription}
\item[\code{address}] est un objet de type \code{character} indiquant le dossier dans lequel se situe l'ensemble des donnees necessaires
pour la construction de l'objet.
\end{ldescription}
\end{Arguments}
%
\begin{Details}\relax
La creation d'un objet \code{\LinkA{Obligation}{Obligation}} necessite des donnees presentes dans un fichier nomme \code{Obligations.csv}.
\end{Details}
%
\begin{Author}\relax
Damien Tichit pour Sia Partners
\end{Author}
\inputencoding{utf8}
\HeaderA{load\_passif}{Fonction \code{load\_passif}.}{load.Rul.passif}
%
\begin{Description}\relax
Cette fonction permet de charger les donnees pour un objet de type \code{\LinkA{Passif}{Passif}}. Les donnees auront ete prealablement disposees dans
une architecture propre a \code{SiALM}.
\end{Description}
%
\begin{Usage}
\begin{verbatim}
load_passif(address)
\end{verbatim}
\end{Usage}
%
\begin{Arguments}
\begin{ldescription}
\item[\code{address}] est un objet de type \code{character} indiquant le dossier dans lequel se situe l'ensemble des donnees necessaires
pour la construction de l'objet.
\end{ldescription}
\end{Arguments}
%
\begin{Details}\relax
La creation d'un objet \code{\LinkA{Passif}{Passif}} necessite la creation de deux autres objets : \code{\LinkA{Passif}{Passif}}, \code{\LinkA{Provision}{Provision}} et \code{\LinkA{HypPassif}{HypPassif}}.
\end{Details}
%
\begin{Author}\relax
Damien Tichit pour Sia Partners
\end{Author}
%
\begin{SeeAlso}\relax
Construction d'un objet de type \code{\LinkA{PTFPassif}{PTFPassif}} : \code{\LinkA{load\_ptf\_passif}{load.Rul.ptf.Rul.passif}}.

Construction d'un objet de type \code{\LinkA{HypPassif}{HypPassif}} : \code{\LinkA{load\_hyp\_passif}{load.Rul.hyp.Rul.passif}}.

Construction d'un objet de type \code{\LinkA{Provision}{Provision}} : \code{\LinkA{load\_provision}{load.Rul.provision}}.
\end{SeeAlso}
\inputencoding{utf8}
\HeaderA{load\_ppe}{Fonction \code{load\_ppe}.}{load.Rul.ppe}
%
\begin{Description}\relax
Cette fonction permet de charger les donnees pour un objet de type \code{\LinkA{PPE}{PPE}}. Les donnees auront ete prealablement disposees dans
une architecture propre a \code{SiALM}.
\end{Description}
%
\begin{Usage}
\begin{verbatim}
load_ppe(address)
\end{verbatim}
\end{Usage}
%
\begin{Arguments}
\begin{ldescription}
\item[\code{address}] est un objet de type \code{character} indiquant le dossier dans lequel se situe l'ensemble des donnees necessaires
pour la construction de l'objet.
\end{ldescription}
\end{Arguments}
%
\begin{Details}\relax
La creation d'un objet \code{\LinkA{PPE}{PPE}} necessite des donnees presentes dans les fichiers nommes \code{PPE.csv} et \code{TM-H.csv}.
\end{Details}
%
\begin{Author}\relax
Damien Tichit pour Sia Partners
\end{Author}
\inputencoding{utf8}
\HeaderA{load\_pre}{Fonction \code{load\_pre}.}{load.Rul.pre}
%
\begin{Description}\relax
Cette fonction permet de charger les donnees pour un objet de type \code{\LinkA{PRE}{PRE}}. Les donnees auront ete prealablement disposees dans
une architecture propre a \code{SiALM}.
\end{Description}
%
\begin{Usage}
\begin{verbatim}
load_pre(address)
\end{verbatim}
\end{Usage}
%
\begin{Arguments}
\begin{ldescription}
\item[\code{address}] est un objet de type \code{character} indiquant le dossier dans lequel se situe l'ensemble des donnees necessaires
pour la construction de l'objet.
\end{ldescription}
\end{Arguments}
%
\begin{Details}\relax
La creation d'un objet \code{\LinkA{PRE}{PRE}} necessite des donnees presentes dans les fichiers nommes \code{pre.csv}.
\end{Details}
%
\begin{Author}\relax
Damien Tichit pour Sia Partners
\end{Author}
\inputencoding{utf8}
\HeaderA{load\_provision}{Fonction \code{load\_provision}.}{load.Rul.provision}
%
\begin{Description}\relax
Cette fonction permet de charger les donnees pour un objet de type \code{\LinkA{Provision}{Provision}}.
Les donnees auront ete prealablement disposees dans une architecture propre a \code{SiALM}.
\end{Description}
%
\begin{Usage}
\begin{verbatim}
load_provision(address)
\end{verbatim}
\end{Usage}
%
\begin{Arguments}
\begin{ldescription}
\item[\code{address}] est un objet de type \code{character} indiquant le dossier dans lequel se situe l'ensemble des donnees necessaires
pour la construction de l'objet.
\end{ldescription}
\end{Arguments}
%
\begin{Details}\relax
La creation d'un objet \code{\LinkA{Provision}{Provision}} necessite la creation de deux autres objets : \code{\LinkA{PPE}{PPE}} et \code{\LinkA{ReserveCapi}{ReserveCapi}}.
\end{Details}
%
\begin{Author}\relax
Damien Tichit pour Sia Partners
\end{Author}
%
\begin{SeeAlso}\relax
Construction d'un objet de type \code{\LinkA{PPE}{PPE}} : \code{\LinkA{load\_ppe}{load.Rul.ppe}}.

Construction d'un objet de type \code{\LinkA{ReserveCapi}{ReserveCapi}} : \code{\LinkA{load\_reserve\_capi}{load.Rul.reserve.Rul.capi}}.
\end{SeeAlso}
\inputencoding{utf8}
\HeaderA{load\_ptf\_actif}{Fonction \code{load\_ptf\_actif}.}{load.Rul.ptf.Rul.actif}
%
\begin{Description}\relax
Cette fonction permet de charger les donnees pour un objet de type \code{\LinkA{Actif}{Actif}}. Les donnees auront ete prealablement disposees dans
une architecture propre a \code{SiALM}.
\end{Description}
%
\begin{Usage}
\begin{verbatim}
load_ptf_actif(address)
\end{verbatim}
\end{Usage}
%
\begin{Arguments}
\begin{ldescription}
\item[\code{address}] est un objet de type \code{character} indiquant le dossier dans lequel se situe l'ensemble des donnees necessaires
pour la construction de l'objet.
\end{ldescription}
\end{Arguments}
%
\begin{Details}\relax
La creation d'un objet \code{\LinkA{Actif}{Actif}} necessite la creation de trois autres objets : \code{\LinkA{Action}{Action}}, \code{\LinkA{Obligation}{Obligation}}, \code{\LinkA{Immobilier}{Immobilier}} et \code{\LinkA{Tresorerie}{Tresorerie}}.
\end{Details}
%
\begin{Author}\relax
Damien Tichit pour Sia Partners
\end{Author}
%
\begin{SeeAlso}\relax
Construction d'un objet de type \code{\LinkA{Action}{Action}} : \code{\LinkA{load\_action}{load.Rul.action}}.

Construction d'un objet de type \code{\LinkA{Obligation}{Obligation}} : \code{\LinkA{load\_obligation}{load.Rul.obligation}}.

Construction d'un objet de type \code{\LinkA{Tresorerie}{Tresorerie}} : \code{\LinkA{load\_tresorerie}{load.Rul.tresorerie}}.

Construction d'un objet de type \code{\LinkA{Immobilier}{Immobilier}} : \code{\LinkA{load\_immobilier}{load.Rul.immobilier}}.
\end{SeeAlso}
\inputencoding{utf8}
\HeaderA{load\_ptf\_cible}{Fonction \code{load\_ptf\_cible}.}{load.Rul.ptf.Rul.cible}
%
\begin{Description}\relax
Cette fonction permet de charger les donnees pour un objet de type \code{\LinkA{PTFCible}{PTFCible}}. Les donnees auront ete prealablement disposees dans
une architecture propre a \code{SiALM}.
\end{Description}
%
\begin{Usage}
\begin{verbatim}
load_ptf_cible(address)
\end{verbatim}
\end{Usage}
%
\begin{Arguments}
\begin{ldescription}
\item[\code{address}] est un objet de type \code{character} indiquant le dossier dans lequel se situe l'ensemble des donnees necessaires
pour la construction de l'objet.
\end{ldescription}
\end{Arguments}
%
\begin{Details}\relax
La creation d'un objet \code{\LinkA{PTFCible}{PTFCible}} necessite la creation d'un autre objet : \code{\LinkA{Obligation}{Obligation}}.
\end{Details}
%
\begin{Author}\relax
Damien Tichit pour Sia Partners
\end{Author}
%
\begin{SeeAlso}\relax
Construction d'un objet de type \code{\LinkA{Obligation}{Obligation}} : \code{\LinkA{load\_obligation\_cible}{load.Rul.obligation.Rul.cible}}.
\end{SeeAlso}
\inputencoding{utf8}
\HeaderA{load\_ptf\_passif}{Fonction \code{load\_ptf\_passif}.}{load.Rul.ptf.Rul.passif}
%
\begin{Description}\relax
Cette fonction permet de charger les donnees pour un objet de type \code{\LinkA{PTFPassif}{PTFPassif}}. Les donnees auront ete prealablement disposees dans
une architecture propre a \code{SiALM}.
\end{Description}
%
\begin{Usage}
\begin{verbatim}
load_ptf_passif(address)
\end{verbatim}
\end{Usage}
%
\begin{Arguments}
\begin{ldescription}
\item[\code{address}] est un objet de type \code{character} indiquant le dossier dans lequel se situe l'ensemble des donnees necessaires
pour la construction de l'objet.
\end{ldescription}
\end{Arguments}
%
\begin{Details}\relax
La creation d'un objet \code{\LinkA{PTFPassif}{PTFPassif}} necessite la creation d'un autre objet : \code{\LinkA{Epargne}{Epargne}}.
\end{Details}
%
\begin{Author}\relax
Damien Tichit pour Sia Partners
\end{Author}
%
\begin{SeeAlso}\relax
Construction d'un objet de type \code{\LinkA{Epargne}{Epargne}} : \code{\LinkA{load\_epargne}{load.Rul.epargne}}.
\end{SeeAlso}
\inputencoding{utf8}
\HeaderA{load\_rachat\_conj}{Fonction \code{load\_rachat\_conj}.}{load.Rul.rachat.Rul.conj}
%
\begin{Description}\relax
Cette fonction permet de charger les donnees pour un objet de type \code{\LinkA{RachatConj}{RachatConj}}. Les donnees auront ete prealablement disposees dans
une architecture propre a \code{SiALM}.
\end{Description}
%
\begin{Usage}
\begin{verbatim}
load_rachat_conj(address)
\end{verbatim}
\end{Usage}
%
\begin{Arguments}
\begin{ldescription}
\item[\code{address}] est un objet de type \code{character} indiquant le dossier dans lequel se situe l'ensemble des donnees necessaires
pour la construction de l'objet.
\end{ldescription}
\end{Arguments}
%
\begin{Details}\relax
La creation d'un objet \code{\LinkA{RachatConj}{RachatConj}} necessite des donnees presentes dans un fichier nommes \code{Rachats\_conjoncturels.csv}.
\end{Details}
%
\begin{Author}\relax
Damien Tichit pour Sia Partners
\end{Author}
\inputencoding{utf8}
\HeaderA{load\_reserve\_capi}{Fonction \code{load\_reserve\_capi}.}{load.Rul.reserve.Rul.capi}
%
\begin{Description}\relax
Cette fonction permet de charger les donnees pour un objet de type \code{\LinkA{ReserveCapi}{ReserveCapi}}. Les donnees auront ete prealablement disposees dans
une architecture propre a \code{SiALM}.
\end{Description}
%
\begin{Usage}
\begin{verbatim}
load_reserve_capi(address)
\end{verbatim}
\end{Usage}
%
\begin{Arguments}
\begin{ldescription}
\item[\code{address}] est un objet de type \code{character} indiquant le dossier dans lequel se situe l'ensemble des donnees necessaires
pour la construction de l'objet.
\end{ldescription}
\end{Arguments}
%
\begin{Details}\relax
La creation d'un objet \code{\LinkA{ReserveCapi}{ReserveCapi}} necessite des donnees presentes dans les fichiers nommes \code{reserve\_capi.csv}.
\end{Details}
%
\begin{Author}\relax
Damien Tichit pour Sia Partners
\end{Author}
\inputencoding{utf8}
\HeaderA{load\_system}{Fonction \code{load\_system}.}{load.Rul.system}
%
\begin{Description}\relax
Cette fonction permet de charger les donnees pour un objet de type \code{\LinkA{System}{System}}. Les donnees auront ete prealablement disposees dans
une architecture propre a \code{SiALM}.
\end{Description}
%
\begin{Usage}
\begin{verbatim}
load_system(address)
\end{verbatim}
\end{Usage}
%
\begin{Arguments}
\begin{ldescription}
\item[\code{address}] est un objet de type \code{character} indiquant le dossier dans lequel se situe l'ensemble des donnees necessaires
pour la construction de l'objet.
\end{ldescription}
\end{Arguments}
%
\begin{Details}\relax
La creation d'un objet \code{\LinkA{System}{System}} necessite la creation de deux autres objets : \code{\LinkA{Actif}{Actif}} et \code{\LinkA{Passif}{Passif}}.
\end{Details}
%
\begin{Author}\relax
Damien Tichit pour Sia Partners
\end{Author}
%
\begin{SeeAlso}\relax
Construction d'un objet de type \code{\LinkA{Actif}{Actif}} : \code{\LinkA{load\_actif}{load.Rul.actif}}.

Construction d'un objet de type \code{\LinkA{Passif}{Passif}} : \code{\LinkA{load\_passif}{load.Rul.passif}}.
\end{SeeAlso}
\inputencoding{utf8}
\HeaderA{load\_tab\_morta}{Fonction \code{load\_tab\_morta}.}{load.Rul.tab.Rul.morta}
%
\begin{Description}\relax
Cette fonction permet de charger les donnees pour un objet de type \code{\LinkA{TabMorta}{TabMorta}}. Les donnees auront ete prealablement disposees dans
une architecture propre a \code{SiALM}.
\end{Description}
%
\begin{Usage}
\begin{verbatim}
load_tab_morta(address)
\end{verbatim}
\end{Usage}
%
\begin{Arguments}
\begin{ldescription}
\item[\code{address}] est un objet de type \code{character} indiquant le dossier dans lequel se situe l'ensemble des donnees necessaires
pour la construction de l'objet.
\end{ldescription}
\end{Arguments}
%
\begin{Details}\relax
La creation d'un objet \code{\LinkA{TabMorta}{TabMorta}} necessite des donnees presentes dans les fichiers nommes \code{TM-F.csv} et \code{TM-H.csv}.
\end{Details}
%
\begin{Author}\relax
Damien Tichit pour Sia Partners
\end{Author}
\inputencoding{utf8}
\HeaderA{load\_tab\_rachat}{Fonction \code{load\_tab\_rachat}.}{load.Rul.tab.Rul.rachat}
%
\begin{Description}\relax
Cette fonction permet de charger les donnees pour un objet de type \code{\LinkA{TabRachat}{TabRachat}}. Les donnees auront ete prealablement disposees dans
une architecture propre a \code{SiALM}.
\end{Description}
%
\begin{Usage}
\begin{verbatim}
load_tab_rachat(address)
\end{verbatim}
\end{Usage}
%
\begin{Arguments}
\begin{ldescription}
\item[\code{address}] est un objet de type \code{character} indiquant le dossier dans lequel se situe l'ensemble des donnees necessaires
pour la construction de l'objet.
\end{ldescription}
\end{Arguments}
%
\begin{Details}\relax
La creation d'un objet \code{\LinkA{TabRachat}{TabRachat}} necessite des donnees presentes dans un fichier nomme \code{Rachat.csv}.
\end{Details}
%
\begin{Author}\relax
Damien Tichit pour Sia Partners
\end{Author}
\inputencoding{utf8}
\HeaderA{load\_tresorerie}{Fonction \code{load\_tresorerie}.}{load.Rul.tresorerie}
%
\begin{Description}\relax
Cette fonction permet de charger les donnees pour un objet de type \code{\LinkA{Tresorerie}{Tresorerie}}. Les donnees auront ete prealablement disposees dans
une architecture propre a \code{SiALM}.
\end{Description}
%
\begin{Usage}
\begin{verbatim}
load_tresorerie(address)
\end{verbatim}
\end{Usage}
%
\begin{Arguments}
\begin{ldescription}
\item[\code{address}] est un objet de type \code{character} indiquant le dossier dans lequel se situe l'ensemble des donnees necessaires
pour la construction de l'objet.
\end{ldescription}
\end{Arguments}
%
\begin{Details}\relax
La creation d'un objet \code{\LinkA{Tresorerie}{Tresorerie}} necessite des donnees presentes dans un fichier nomme \code{Tresorerie.csv}.
\end{Details}
%
\begin{Author}\relax
Damien Tichit pour Sia Partners
\end{Author}
\inputencoding{utf8}
\HeaderA{Obligation}{Classe \code{Obligation}}{Obligation}
\keyword{classes}{Obligation}
%
\begin{Description}\relax
Cette classe represente le portefeuille des obligations de la compagnie d'assurance.
\end{Description}
%
\begin{Section}{Slots}

\begin{description}

\item[\code{ptf}] est un objet de type \code{\LinkA{data.frame}{data.frame}} contenant les donnees relatives au portfeuille.

\end{description}
\end{Section}
%
\begin{Author}\relax
Damien Tichit pour Sia Partners
\end{Author}
\inputencoding{utf8}
\HeaderA{Passif}{Classe \code{Passif}}{Passif}
\keyword{classes}{Passif}
%
\begin{Description}\relax
Cette classe aggrege l'ensemble des donnees relatives au passif de la compagnie d'assurance : hypotheses, portefeuille, provisions
\end{Description}
%
\begin{Section}{Slots}

\begin{description}

\item[\code{ptf\_passif}] est un objet de la classe \code{\LinkA{PTFPassif}{PTFPassif}} representant le portfeuille passif.

\item[\code{hyp\_passif}] est un objet de la classe \code{\LinkA{HypPassif}{HypPassif}} contenant l'ensemble des hypotheses du passif.

\item[\code{provision}] est un objet de la classe \code{\LinkA{Provision}{Provision}}.

\item[\code{fonds\_propres}] est un objet de la classe \code{\LinkA{FondsPropres}{FondsPropres}}.

\end{description}
\end{Section}
%
\begin{Author}\relax
Damien Tichit pour Sia Partners
\end{Author}
\inputencoding{utf8}
\HeaderA{pna.omit}{Fonction \code{pna.omit}}{pna.omit}
%
\begin{Description}\relax
Cette fonction, prenant deux vecteurs, sort un seul vecteur sans NAs.
\end{Description}
%
\begin{Usage}
\begin{verbatim}
pna.omit(x, y)
\end{verbatim}
\end{Usage}
%
\begin{Arguments}
\begin{ldescription}
\item[\code{x}] premier vecteur.

\item[\code{y}] second vecteur.
\end{ldescription}
\end{Arguments}
%
\begin{Author}\relax
Damien Tichit pour Sia Partners
\end{Author}
\inputencoding{utf8}
\HeaderA{PPE}{Classe \code{PPE}}{PPE}
\keyword{classes}{PPE}
%
\begin{Description}\relax
Cette classe represente le Provision pour Participation aux Excedents.
\end{Description}
%
\begin{Section}{Slots}

\begin{description}

\item[\code{ppe}] est un \code{\LinkA{numeric}{numeric}} contenant les montants dotes sur les huits dernieres annees.

\end{description}
\end{Section}
%
\begin{Author}\relax
Damien Tichit pour Sia Partners
\end{Author}
\inputencoding{utf8}
\HeaderA{PRE}{Classe \code{PRE}}{PRE}
\keyword{classes}{PRE}
%
\begin{Description}\relax
Cette classe permet de modeliser la PRE.
\end{Description}
%
\begin{Section}{Slots}

\begin{description}

\item[\code{montant}] est un \code{numeric}.

\end{description}
\end{Section}
%
\begin{Author}\relax
Damien Tichit pour Sia Partners
\end{Author}
\inputencoding{utf8}
\HeaderA{ProbaEpargne}{Classe \code{ProbaEpargne}}{ProbaEpargne}
\keyword{classes}{ProbaEpargne}
%
\begin{Description}\relax
Cette classe permet de stocker les differentes probas relatives a un portfeuille epargne : deces et rachats.
\end{Description}
%
\begin{Section}{Slots}

\begin{description}

\item[\code{deces\_contr}] est un objet de type \code{data.frame}.

\item[\code{deces\_pm}] est un objet de type \code{data.frame}.

\item[\code{rachat\_part}] est un objet de type \code{data.frame}.

\item[\code{rachat\_tot\_contr}] est un objet de type \code{data.frame}.

\item[\code{rachat\_tot\_pm}] est un objet de type \code{data.frame}.

\end{description}
\end{Section}
%
\begin{Author}\relax
Damien Tichit pour Sia Partners
\end{Author}
\inputencoding{utf8}
\HeaderA{proj\_1an\_actif}{Fonction \code{proj\_1an\_actif}.}{proj.Rul.1an.Rul.actif}
%
\begin{Description}\relax
Cette fonction permet de projeter horizon 1 an l'actif d'une compagnie d'assurance.
\end{Description}
%
\begin{Usage}
\begin{verbatim}
proj_1an_actif(actif, an)
\end{verbatim}
\end{Usage}
%
\begin{Arguments}
\begin{ldescription}
\item[\code{actif}] est un objet de type \code{\LinkA{Actif}{Actif}}.

\item[\code{an}] est un \code{integer} reprensentant l'annee sur laquelle on travaille.
\end{ldescription}
\end{Arguments}
%
\begin{Author}\relax
Damien Tichit pour Sia Partners
\end{Author}
\inputencoding{utf8}
\HeaderA{proj\_1an\_epargne}{Fonction \code{proj\_1an\_epargne}.}{proj.Rul.1an.Rul.epargne}
%
\begin{Description}\relax
Cette fonction permet de projeter horizon 1 an un portefeuille de contrats epragnes.
\end{Description}
%
\begin{Usage}
\begin{verbatim}
proj_1an_epargne(epargne, hyp_passif, an)
\end{verbatim}
\end{Usage}
%
\begin{Arguments}
\begin{ldescription}
\item[\code{epargne}] est un objet de type \code{\LinkA{Epargne}{Epargne}}.

\item[\code{hyp\_passif}] est un objet de type \code{\LinkA{HypPassif}{HypPassif}}.

\item[\code{an}] est un objet de type \code{integer}.
\end{ldescription}
\end{Arguments}
%
\begin{Author}\relax
Damien Tichit pour Sia Partners
\end{Author}
\inputencoding{utf8}
\HeaderA{proj\_1an\_passif}{Fonction \code{proj\_1an\_passif}}{proj.Rul.1an.Rul.passif}
%
\begin{Description}\relax
Cette fonction permet de projeter horizon 1 an le passif d'une compagnie d'assurance.
\end{Description}
%
\begin{Usage}
\begin{verbatim}
proj_1an_passif(passif, an)
\end{verbatim}
\end{Usage}
%
\begin{Arguments}
\begin{ldescription}
\item[\code{passif}] est un objet de type \code{\LinkA{Passif}{Passif}}.

\item[\code{an}] est un objet de type \code{integer}.
\end{ldescription}
\end{Arguments}
%
\begin{Author}\relax
Damien Tichit pour Sia Partners
\end{Author}
\inputencoding{utf8}
\HeaderA{proj\_1an\_ptf\_passif}{Fonction \code{proj\_1an\_ptf\_passif}}{proj.Rul.1an.Rul.ptf.Rul.passif}
%
\begin{Description}\relax
Cette fonction permet de projeter horizon 1 an le portfeuille passif : gestion des differents passifs
\end{Description}
%
\begin{Usage}
\begin{verbatim}
proj_1an_ptf_passif(ptf_passif, hyp_passif, an)
\end{verbatim}
\end{Usage}
%
\begin{Arguments}
\begin{ldescription}
\item[\code{ptf\_passif}] est un objet de type \code{\LinkA{PTFPassif}{PTFPassif}}.

\item[\code{hyp\_passif}] est un objet de type \code{\LinkA{HypPassif}{HypPassif}}.

\item[\code{an}] est un objet de type \code{integer}.
\end{ldescription}
\end{Arguments}
%
\begin{Author}\relax
Damien Tichit pour Sia Partners
\end{Author}
\inputencoding{utf8}
\HeaderA{proj\_1an\_system}{Fonction \code{proj\_1an\_system}.}{proj.Rul.1an.Rul.system}
%
\begin{Description}\relax
Cette fonction permet de projeter a horizon 1 an un objet \code{\LinkA{System}{System}}.
\end{Description}
%
\begin{Usage}
\begin{verbatim}
proj_1an_system(system, an)
\end{verbatim}
\end{Usage}
%
\begin{Arguments}
\begin{ldescription}
\item[\code{system}] est un objet de type \code{System}.

\item[\code{an}] est un \code{integer}.
\end{ldescription}
\end{Arguments}
%
\begin{Author}\relax
Damien Tichit pour Sia Partners
\end{Author}
%
\begin{SeeAlso}\relax
Projection des passifs : \code{\LinkA{proj\_1an\_passif}{proj.Rul.1an.Rul.passif}}

Projection des actifs : \code{\LinkA{proj\_1an\_actif}{proj.Rul.1an.Rul.actif}}
\end{SeeAlso}
\inputencoding{utf8}
\HeaderA{Provision}{Classe \code{Provision}}{Provision}
\keyword{classes}{Provision}
%
\begin{Description}\relax
Cette classe aggrege les differentes provisions relatives au passif d'une compagnie d'assurance : PPE, Reserve de Capitalisation
\end{Description}
%
\begin{Section}{Slots}

\begin{description}

\item[\code{ppe}] est un objet de la classe \code{\LinkA{PPE}{PPE}}.

\item[\code{reserve\_capi}] est un objet de la classe \code{\LinkA{ReserveCapi}{ReserveCapi}}.

\item[\code{pre}] est un objet de la classe \code{\LinkA{PRE}{PRE}}.

\end{description}
\end{Section}
%
\begin{Author}\relax
Damien Tichit pour Sia Partners
\end{Author}
\inputencoding{utf8}
\HeaderA{psum}{Fonction \code{psum}}{psum}
%
\begin{Description}\relax
Cette fonction permet de faire la somme, sur plusieurs vecteurs, element par element, et pouvant supprimer les NAs.
\end{Description}
%
\begin{Usage}
\begin{verbatim}
psum(..., na.rm = FALSE)
\end{verbatim}
\end{Usage}
%
\begin{Arguments}
\begin{ldescription}
\item[\code{...}] les differnts vecteurs.

\item[\code{na.rm}] est un \code{logical} qui, lorsqu'il est a TRUE, permet a la somme de ne pas prendre en compte les NAs.
\end{ldescription}
\end{Arguments}
%
\begin{Author}\relax
Damien Tichit pour Sia Partners
\end{Author}
\inputencoding{utf8}
\HeaderA{PTFActif}{Classe \code{PTFActif}}{PTFActif}
\keyword{classes}{PTFActif}
%
\begin{Description}\relax
Cette classe represente le portfeuille financier de la compagnie d'assurance.
\end{Description}
%
\begin{Section}{Slots}

\begin{description}

\item[\code{action}] est un objet de la classe \code{\LinkA{Action}{Action}} representant le portfeuille action.

\item[\code{obligation}] est un objet de la classe \code{\LinkA{Obligation}{Obligation}} representant le portfeuille obligation.

\item[\code{tresorerie}] est un objet de la classe \code{\LinkA{Tresorerie}{Tresorerie}} representant le portfeuille tresorerie.

\item[\code{immobilier}] est un objet de la classe \code{\LinkA{Immobilier}{Immobilier}} representant le portfeuille tresorerie.

\end{description}
\end{Section}
%
\begin{Author}\relax
Damien Tichit pour Sia Partners
\end{Author}
\inputencoding{utf8}
\HeaderA{PTFCible}{Classe \code{PTFCible}}{PTFCible}
\keyword{classes}{PTFCible}
%
\begin{Description}\relax
Cette classe represente le portfeuille financier cible de la compagnie d'assurance dans le cadre de
\end{Description}
%
\begin{Section}{Slots}

\begin{description}

\item[\code{obligation}] est un objet de la classe \code{\LinkA{Obligation}{Obligation}} representant le portfeuille cible obligation.

\item[\code{alloc\_cible}] est un objet \code{data.frame} indiquant l'allocation cible pour chaque produit.

\end{description}
\end{Section}
%
\begin{Author}\relax
Damien Tichit pour Sia Partners
\end{Author}
\inputencoding{utf8}
\HeaderA{PTFPassif}{Classe \code{PTFPassif}}{PTFPassif}
\keyword{classes}{PTFPassif}
%
\begin{Description}\relax
Cette classe represente le portfeuille passif de la compagnie d'assurance.
\end{Description}
%
\begin{Section}{Slots}

\begin{description}

\item[\code{epargne}] est un objet de la classe \code{\LinkA{Epargne}{Epargne}} representant le portfeuille epargne.

\end{description}
\end{Section}
%
\begin{Author}\relax
Damien Tichit pour Sia Partners
\end{Author}
\inputencoding{utf8}
\HeaderA{RachatConj}{Classe \code{RachatConj}}{RachatConj}
\keyword{classes}{RachatConj}
%
\begin{Description}\relax
Cette classe permet de modeliser les rachats conjoncturels.
\end{Description}
%
\begin{Section}{Slots}

\begin{description}

\item[\code{alpha}] est un \code{numeric}

\item[\code{beta}] est un \code{numeric}

\item[\code{gamma}] est un \code{numeric}

\item[\code{delta}] est un \code{numeric}

\item[\code{RCmin}] est un \code{numeric}

\item[\code{RCmax}] est un \code{numeric}

\item[\code{repartition}] est une \code{list}

\end{description}
\end{Section}
%
\begin{Author}\relax
Damien Tichit pour Sia Partners
\end{Author}
\inputencoding{utf8}
\HeaderA{realisation\_pvl\_action}{Fonction \code{realisation\_pvl\_action}}{realisation.Rul.pvl.Rul.action}
%
\begin{Description}\relax
Cette fonction permet de realiser des plus values lattentes.
\end{Description}
%
\begin{Usage}
\begin{verbatim}
realisation_pvl_action(action, montant)
\end{verbatim}
\end{Usage}
%
\begin{Arguments}
\begin{ldescription}
\item[\code{action}] est un objet de type \code{Action}.

\item[\code{montant}] est un \code{numeric} indiquant le montant de plus value devant etre realise.
\end{ldescription}
\end{Arguments}
%
\begin{Author}\relax
Damien Tichit pour Sia Partners
\end{Author}
\inputencoding{utf8}
\HeaderA{realisation\_pvl\_immobilier}{Fonction \code{realisation\_pvl\_immobilier}}{realisation.Rul.pvl.Rul.immobilier}
%
\begin{Description}\relax
Cette fonction permet de realiser des plus values lattentes.
\end{Description}
%
\begin{Usage}
\begin{verbatim}
realisation_pvl_immobilier(immobilier, montant)
\end{verbatim}
\end{Usage}
%
\begin{Arguments}
\begin{ldescription}
\item[\code{immobilier}] est un objet de type \code{Immobilier}.

\item[\code{montant}] est un \code{numeric} indiquant le montant de plus value devant etre realise.
\end{ldescription}
\end{Arguments}
%
\begin{Author}\relax
Damien Tichit pour Sia Partners
\end{Author}
\inputencoding{utf8}
\HeaderA{realisation\_pvl\_obligation}{Fonction \code{realisation\_pvl\_obligation}}{realisation.Rul.pvl.Rul.obligation}
%
\begin{Description}\relax
Cette fonction permet de realiser des plus values lattentes pour un portefeuille obligataire.
\end{Description}
%
\begin{Usage}
\begin{verbatim}
realisation_pvl_obligation(obligation, montant)
\end{verbatim}
\end{Usage}
%
\begin{Arguments}
\begin{ldescription}
\item[\code{obligation}] est un objet de type \code{Obligation}.

\item[\code{montant}] est un \code{numeric} indiquant le montant de plus value devant etre realise.
\end{ldescription}
\end{Arguments}
%
\begin{Author}\relax
Damien Tichit pour Sia Partners
\end{Author}
\inputencoding{utf8}
\HeaderA{realisation\_pvl\_ptf\_actif}{Fonction \code{realisation\_pvl\_ptf\_actif}}{realisation.Rul.pvl.Rul.ptf.Rul.actif}
%
\begin{Description}\relax
Cette fonction permet de realiser des plus values lattentes.
\end{Description}
%
\begin{Usage}
\begin{verbatim}
realisation_pvl_ptf_actif(ptf_actif, montant)
\end{verbatim}
\end{Usage}
%
\begin{Arguments}
\begin{ldescription}
\item[\code{ptf\_actif}] est un objet de type \code{PTFActif}.

\item[\code{montant}] est un \code{numeric} indiquant le montant de plus value devant etre realise.
\end{ldescription}
\end{Arguments}
%
\begin{Author}\relax
Damien Tichit pour Sia Partners
\end{Author}
\inputencoding{utf8}
\HeaderA{rebalancement\_actif}{Fonction \code{rebalancement\_actif}}{rebalancement.Rul.actif}
%
\begin{Description}\relax
Cette fonction permet de rebalancer le portefeuille d'actif vers le portfeuile cible.
\end{Description}
%
\begin{Usage}
\begin{verbatim}
rebalancement_actif(actif, solde_tresorerie = 0)
\end{verbatim}
\end{Usage}
%
\begin{Arguments}
\begin{ldescription}
\item[\code{actif}] est un objet de type \code{Actif}.

\item[\code{solde\_tresorerie}] est un \code{numeric} represantant le montant du solde de tresorerie. Par defaut, cette valeur est egale a 0.
\end{ldescription}
\end{Arguments}
%
\begin{Author}\relax
Damien Tichit pour Sia Partners
\end{Author}
\inputencoding{utf8}
\HeaderA{rebalancement\_action}{Fonction \code{rebalancement\_action}}{rebalancement.Rul.action}
%
\begin{Description}\relax
Cette fonction permet de rebalancer le portfeuille d'action vers un portfeuile cible.
Le montant total cible, en valeur de marche, du portefeuille cible est renseigne dans le parametre \code{alloc\_cible}.
\end{Description}
%
\begin{Usage}
\begin{verbatim}
rebalancement_action(action, alloc_cible)
\end{verbatim}
\end{Usage}
%
\begin{Arguments}
\begin{ldescription}
\item[\code{action}] est un objet de type \code{\LinkA{Action}{Action}}. Ce parametre represente le ptf actuel de la compagnie.

\item[\code{alloc\_cible}] est un \code{numeric}. Ce parametre indique l'allocation cible a atteindre (en valeur de marche).
\end{ldescription}
\end{Arguments}
%
\begin{Author}\relax
Damien Tichit pour Sia Partners
\end{Author}
\inputencoding{utf8}
\HeaderA{rebalancement\_immobilier}{Fonction \code{rebalancement\_immobilier}}{rebalancement.Rul.immobilier}
%
\begin{Description}\relax
Cette fonction permet de rebalancer le portfeuille immobilier vers un portfeuile cible.
Le montant total, en valeur de marche, du portefeuille cible est renseigne dans le parametre \code{alloc\_cible}.
\end{Description}
%
\begin{Usage}
\begin{verbatim}
rebalancement_immobilier(immo, alloc_cible)
\end{verbatim}
\end{Usage}
%
\begin{Arguments}
\begin{ldescription}
\item[\code{immo}] est un objet de type \code{\LinkA{Immobilier}{Immobilier}}. Ce parametre represente le ptf actuel de la compagnie.

\item[\code{alloc\_cible}] est un \code{numeric}. Ce parametre indique l'allocation cible a atteindre.

\item[\code{immo\_cible}] est un objet de type \code{\LinkA{Immobilier}{Immobilier}}. Ce parametre represente le ptf cible.
\end{ldescription}
\end{Arguments}
%
\begin{Author}\relax
Damien Tichit pour Sia Partners
\end{Author}
\inputencoding{utf8}
\HeaderA{rebalancement\_obligation}{Fonction \code{rebalancement\_obligation}}{rebalancement.Rul.obligation}
%
\begin{Description}\relax
Cette fonction permet de rebalancer le portfeuille d'obligation vers un portfeuile cible.
Le montant total, en valeur de marche, du portefeuille cible est renseigne dans le parametre \code{alloc\_cible}.
\end{Description}
%
\begin{Usage}
\begin{verbatim}
rebalancement_obligation(oblig, oblig_cible, alloc_cible)
\end{verbatim}
\end{Usage}
%
\begin{Arguments}
\begin{ldescription}
\item[\code{oblig}] est un objet de type \code{\LinkA{Obligation}{Obligation}}. Ce parametre represente le ptf actuel de la compagnie.

\item[\code{oblig\_cible}] est un objet de type \code{\LinkA{Obligation}{Obligation}}. Ce parametre represente le ptf cible.

\item[\code{alloc\_cible}] est un \code{numeric}. Ce parametre indique l'allocation cible a atteindre.
\end{ldescription}
\end{Arguments}
%
\begin{Author}\relax
Damien Tichit pour Sia Partners
\end{Author}
\inputencoding{utf8}
\HeaderA{reprise\_ppe}{Fonction \code{reprise\_ppe}.}{reprise.Rul.ppe}
%
\begin{Description}\relax
Cette fonction permet de reprendre un montant sur la PPE. Le montant est prioritairement repris sur les plus vieilles dotations.
\end{Description}
%
\begin{Usage}
\begin{verbatim}
reprise_ppe(ppe, montant)
\end{verbatim}
\end{Usage}
%
\begin{Arguments}
\begin{ldescription}
\item[\code{ppe}] est un objet de type \code{\LinkA{PPE}{PPE}}.

\item[\code{montant}] est un \code{numeric} representant le montant a reprendre.
\end{ldescription}
\end{Arguments}
%
\begin{Author}\relax
Damien Tichit pour Sia Partners
\end{Author}
\inputencoding{utf8}
\HeaderA{reprise\_ppe\_8ans}{Fonction \code{reprise\_ppe\_8ans}.}{reprise.Rul.ppe.Rul.8ans}
%
\begin{Description}\relax
Cette fonction permet de reprendre le montant dotee 8 annees auparavant. Elle met egalement a 0 l'element du vecteur correspondant a la PPE dotee 8 ans auparavant.
\end{Description}
%
\begin{Usage}
\begin{verbatim}
reprise_ppe_8ans(ppe)
\end{verbatim}
\end{Usage}
%
\begin{Arguments}
\begin{ldescription}
\item[\code{ppe}] est un objet de type \code{\LinkA{PPE}{PPE}}.
\end{ldescription}
\end{Arguments}
%
\begin{Author}\relax
Damien Tichit pour Sia Partners
\end{Author}
\inputencoding{utf8}
\HeaderA{ReserveCapi}{Classe \code{ReserveCapi}}{ReserveCapi}
\keyword{classes}{ReserveCapi}
%
\begin{Description}\relax
Cette classe represente la Reserve de Capitalisation.
\end{Description}
%
\begin{Section}{Slots}

\begin{description}

\item[\code{montant}] est un \code{numeric} representant le capital present dans la reserve.

\end{description}
\end{Section}
%
\begin{Author}\relax
Damien Tichit pour Sia Partners
\end{Author}
\inputencoding{utf8}
\HeaderA{revalo\_epargne}{Fonction \code{revalo\_epargne}}{revalo.Rul.epargne}
%
\begin{Description}\relax
Cette fonction permet de revaloriser un portefeuille Epargne. La distribution se fait de la faca suivante :
\begin{description}

\item[Revalorisation cible]  : la revalorisation s'effectue de telle faC'on a ce que tous les MP aient la meme revalorisation et en se rapprochant du taux cible.
\item[Revalorisation supplementaire]  : la revalorisation s'effectue au proportionnellement aux PM.

\end{description}

Les besoins reglementaires sont tout d'abord decomptC) du montant devant etre obligatoirement distribue.
\end{Description}
%
\begin{Usage}
\begin{verbatim}
revalo_epargne(epargne, revalo_cible, revalo_supp, cible)
\end{verbatim}
\end{Usage}
%
\begin{Arguments}
\begin{ldescription}
\item[\code{epargne}] est un objet de type \code{\LinkA{Epargne}{Epargne}}.

\item[\code{revalo\_cible}] est un \code{numeric} representant le montant de revalorisation cible a distibuer.

\item[\code{revalo\_supp}] est un \code{numeric} representant le montant de revalorisation supplementaire a distibuer.

\item[\code{cible}] est un \code{numeric} representant un taux cible.
\end{ldescription}
\end{Arguments}
%
\begin{Author}\relax
Damien Tichit pour Sia Partners
\end{Author}
\inputencoding{utf8}
\HeaderA{revalo\_obligation\_cible}{Fonction \code{revalo\_obligation\_cible}}{revalo.Rul.obligation.Rul.cible}
%
\begin{Description}\relax
Cette fonction permet de revaloriser les differentes obligations d'un portefeuille obligataire cible :
calcul des nouveaux coupons et mise a jour de la valeur de marche.
\end{Description}
%
\begin{Usage}
\begin{verbatim}
revalo_obligation_cible(obligation, yield_curve)
\end{verbatim}
\end{Usage}
%
\begin{Arguments}
\begin{ldescription}
\item[\code{obligation}] est un objet de type \code{\LinkA{Obligation}{Obligation}}.

\item[\code{yield\_curve}] est un \code{numeric} contenant les prix zero-coupon.
\end{ldescription}
\end{Arguments}
%
\begin{Author}\relax
Damien Tichit pour Sia Partners
\end{Author}
\inputencoding{utf8}
\HeaderA{revalo\_passif}{Fonction \code{revalo\_passif}}{revalo.Rul.passif}
%
\begin{Description}\relax
Cette fonction permet de revaloriser le passif d'une compagnie d'assurance. Au cours de cette fonction sont effectuees : la regle des 8 ans.
\end{Description}
%
\begin{Usage}
\begin{verbatim}
revalo_passif(passif, resultat, pvl, revalo_prestation, an)
\end{verbatim}
\end{Usage}
%
\begin{Arguments}
\begin{ldescription}
\item[\code{passif}] est un objet de type \code{\LinkA{Passif}{Passif}}.

\item[\code{resultat}] est un \code{numeric}.

\item[\code{pvl}] est une \code{list} contenant les PVL par produits.

\item[\code{revalo\_prestation}] est une \code{list} indiquant des montants de revalorisation obligatoires ayant deja ete distribue. Ils sont consideres comme des besoins de revalorisation contractuels.
Les montants doivent etre mis sous forme de liste et par produit.

\item[\code{an}] est un \code{integer}.
\end{ldescription}
\end{Arguments}
%
\begin{Details}\relax
Les politiques de revalorisation appliquees dans cette fonction sont les suivantes :
\begin{description}

\item[Epargne]  : Atteindre une revalorisation proched du tme afin de minimiser les rachats conjoncturels.

\end{description}

Les besoins reglementaires sont tout d'abord decomptC) du montant devant etre obligatoirement distribue.
\end{Details}
%
\begin{Author}\relax
Damien Tichit pour Sia Partners
\end{Author}
%
\begin{SeeAlso}\relax
Calcul du besoin en revalorisation : \code{\LinkA{besoin\_revalo\_ptf\_passif}{besoin.Rul.revalo.Rul.ptf.Rul.passif}}.

Gestion de la regle des 8 ans sur la PPE : \code{\LinkA{reprise\_ppe\_8ans}{reprise.Rul.ppe.Rul.8ans}}.
\end{SeeAlso}
\inputencoding{utf8}
\HeaderA{revalo\_ptf\_actif}{Fonction \code{revalo\_ptf\_actif}}{revalo.Rul.ptf.Rul.actif}
%
\begin{Description}\relax
Cette fonction permet de projeter de revaloriser le portfeuille financier.
Elle calcule egalement les plus ou moins value latentes (PMVL) engendrees.
\end{Description}
%
\begin{Usage}
\begin{verbatim}
revalo_ptf_actif(ptf_actif, hyp_actif, an)
\end{verbatim}
\end{Usage}
%
\begin{Arguments}
\begin{ldescription}
\item[\code{ptf\_actif}] est un objet de type \code{\LinkA{PTFActif}{PTFActif}}.

\item[\code{hyp\_actif}] est un objet de type \code{\LinkA{HypActif}{HypActif}}.

\item[\code{an}] est un \code{integer} reprensentant l'annee sur laquelle on travaille.
\end{ldescription}
\end{Arguments}
%
\begin{Author}\relax
Damien Tichit pour Sia Partners
\end{Author}
\inputencoding{utf8}
\HeaderA{revalo\_ptf\_cible}{Fonction \code{revalo\_ptf\_cible}}{revalo.Rul.ptf.Rul.cible}
%
\begin{Description}\relax
Cette fonction permet de revaloriser les portfeuilles cibles : mise a jour des valeurs de marche et des
\end{Description}
%
\begin{Usage}
\begin{verbatim}
revalo_ptf_cible(ptf_cible, esg, an)
\end{verbatim}
\end{Usage}
%
\begin{Arguments}
\begin{ldescription}
\item[\code{ptf\_cible}] est un objet de type \code{\LinkA{PTFCible}{PTFCible}}.

\item[\code{esg}] est une \code{list}.

\item[\code{an}] est un \code{integer} reprensentant l'annee sur laquelle on travaille.
\end{ldescription}
\end{Arguments}
%
\begin{Author}\relax
Damien Tichit pour Sia Partners
\end{Author}
\inputencoding{utf8}
\HeaderA{revalo\_ptf\_passif}{Fonction \code{revalo\_ptf\_passif}}{revalo.Rul.ptf.Rul.passif}
%
\begin{Description}\relax
Cette fonction permet de revaloriser les differents PTF passifs d'une compagnie d'assurance :
\end{Description}
%
\begin{Usage}
\begin{verbatim}
revalo_ptf_passif(ptf_passif, revalo_cible, revalo_supp, cible)
\end{verbatim}
\end{Usage}
%
\begin{Arguments}
\begin{ldescription}
\item[\code{ptf\_passif}] est un objet de type \code{\LinkA{PTFPassif}{PTFPassif}}.

\item[\code{revalo\_cible}] est une \code{list} contenant les montants de revalorisation cibles a distibuer par produit.

\item[\code{revalo\_supp}] est une \code{list} contenant les montants de revalorisation supplementaire a distibuer par produit.

\item[\code{cible}] est une \code{list} contenant des elements relatifs a la politique de revalorisation pour les differents passifs.
\end{ldescription}
\end{Arguments}
%
\begin{Author}\relax
Damien Tichit pour Sia Partners
\end{Author}
\inputencoding{utf8}
\HeaderA{SiALM}{SiALM est un package permettant de calculer un best-estimate pour une compagnie d'assurance proposant les produits suivants : contrats d'epargne.}{SiALM}
\aliasA{SiALM-package}{SiALM}{SiALM.Rdash.package}
\aliasA{SiALM-package}{SiALM}{SiALM.Rdash.package}
%
\begin{Description}\relax
SiALM est un package permettant de calculer un best-estimate pour une compagnie d'assurance
proposant les produits suivants : contrats d'epargne.
\end{Description}
%
\begin{Author}\relax
Damien Tichit
\end{Author}
\inputencoding{utf8}
\HeaderA{System}{Classe \code{System}}{System}
\keyword{classes}{System}
%
\begin{Description}\relax
Cette classe regroupe les actifs et les passifs d'une compagnie d'assurance.
\end{Description}
%
\begin{Section}{Slots}

\begin{description}

\item[\code{passif}] est un objet de type \code{\LinkA{Passif}{Passif}}.

\item[\code{actif}] est un objet de type \code{\LinkA{Actif}{Actif}}.

\item[\code{taux\_pb}] est une \code{list} contenant les taux de pb contractuels.

\end{description}
\end{Section}
%
\begin{Author}\relax
Damien Tichit pour Sia Partners
\end{Author}
\inputencoding{utf8}
\HeaderA{TabMorta}{Classe \code{TabMorta}}{TabMorta}
\keyword{classes}{TabMorta}
%
\begin{Description}\relax
Cette classe represente une table de mortalite.
\end{Description}
%
\begin{Section}{Slots}

\begin{description}

\item[\code{table}] est un objet de type \code{\LinkA{data.frame}{data.frame}} contenant la table de mortalite.

\end{description}
\end{Section}
%
\begin{Author}\relax
Damien Tichit pour Sia Partners
\end{Author}
\inputencoding{utf8}
\HeaderA{TabRachat}{Classe \code{TabRachat}}{TabRachat}
\keyword{classes}{TabRachat}
%
\begin{Description}\relax
Cette classe represente une table de rachat
\end{Description}
%
\begin{Section}{Slots}

\begin{description}

\item[\code{table}] est un objet de type \code{\LinkA{data.frame}{data.frame}} contenant la table de rachat

\end{description}
\end{Section}
%
\begin{Author}\relax
Damien Tichit pour Sia Partners
\end{Author}
\inputencoding{utf8}
\HeaderA{Tresorerie}{Classe \code{Tresorerie}}{Tresorerie}
\keyword{classes}{Tresorerie}
%
\begin{Description}\relax
Cette classe represente la tresorerie de la compagnie d'assurance.
\end{Description}
%
\begin{Section}{Slots}

\begin{description}

\item[\code{solde}] est un objet \code{numeric} indiquant le solde.

\end{description}
\end{Section}
%
\begin{Author}\relax
Damien Tichit pour Sia Partners
\end{Author}
\inputencoding{utf8}
\HeaderA{update\_esg}{Fonction \code{update\_esg}.}{update.Rul.esg}
%
\begin{Description}\relax
Cette fonction permet de mettre a jour les donnees de l'ESG dans la classe \code{\LinkA{HypActif}{HypActif}}.
Elle met les donnees a jour pour une simulation.
\end{Description}
%
\begin{Usage}
\begin{verbatim}
update_esg(alm, num_simu)
\end{verbatim}
\end{Usage}
%
\begin{Arguments}
\begin{ldescription}
\item[\code{alm}] est un objet de type \code{\LinkA{ALM}{ALM}}.

\item[\code{num\_simu}] est un \code{integer} representant le numero de simulation sur lequel on travaille.
\end{ldescription}
\end{Arguments}
%
\begin{Author}\relax
Damien Tichit pour Sia Partners
\end{Author}
\inputencoding{utf8}
\HeaderA{update\_tresorerie}{Fonction \code{update\_tresorerie}.}{update.Rul.tresorerie}
%
\begin{Description}\relax
Cette fonction permet de mettre a jour la tresorerie : credit et debit.
\end{Description}
%
\begin{Usage}
\begin{verbatim}
update_tresorerie(tresorerie, montant)
\end{verbatim}
\end{Usage}
%
\begin{Arguments}
\begin{ldescription}
\item[\code{tresorerie}] est un objet de type \code{\LinkA{Tresorerie}{Tresorerie}}

\item[\code{montant}] est un \code{numeric}
\end{ldescription}
\end{Arguments}
%
\begin{Author}\relax
Damien Tichit pour Sia Partners
\end{Author}
\inputencoding{utf8}
\HeaderA{vieillissement\_ppe}{Fonction \code{vieillissement\_ppe}.}{vieillissement.Rul.ppe}
%
\begin{Description}\relax
Cette fonction permet de vieillir d'une annee l'objet \code{\LinkA{PPE}{PPE}}
\end{Description}
%
\begin{Usage}
\begin{verbatim}
vieillissement_ppe(ppe)
\end{verbatim}
\end{Usage}
%
\begin{Arguments}
\begin{ldescription}
\item[\code{ppe}] est un objet de type \code{\LinkA{PPE}{PPE}}.
\end{ldescription}
\end{Arguments}
%
\begin{Author}\relax
Damien Tichit pour Sia Partners
\end{Author}
